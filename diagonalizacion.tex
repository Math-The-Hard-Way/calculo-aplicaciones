\section{Diagonalización}



\begin{definicion}
 $A\in M_{n}$ se dice que es \emph{diagonalizable} si existe una base de $\R^{n}$ que consista de vectores propios de
$A$. 
\end{definicion}

\begin{problema}
 Determine si la matriz $A$ definida en (\ref{no:diagonal}) es diagonalizable.
\end{problema}

\begin{solucion}
 Como vimos en las secciones anteriores, los valores propios de $A$ son $\lam_{1}=2$ y $\lam_{2}=1,$ con respectivos
espacio propios
$$
\ker(A-2I)=\basis{\bm{1 \\ 1 \\ 2}}.
$$
y
  $$
\ker(A-I)=\basis{\bm{1\\0\\2}}.
  $$

  Como cualquier otro vector propio es o bien multiplo de $\bm{1 \\ 1 \\ 2}$ o bien de $\bm{1\\0\\2},$ tendríamos a los
más una conjuto de dos vectores propios linealmente independientes. Sin embargo, cualquier base de $\R^{3}$ debe tener
exactamente 3 vectores propios linealmente independientes. Por tanto $A$ no es diagonalizable.

Podemos comprobar este resultado usando \texttt{WxMaxima} de la siguiente manera.

Primero, introducimos la matriz de manera habitual.


\noindent
%%%%%%%%%%%%%%%
%%% INPUT:
\begin{minipage}{8ex}{\color{red}\bf
\begin{verbatim}
(%i1)
\end{verbatim}}
\end{minipage}
\begin{minipage}{\textwidth}{\color{blue}
\begin{verbatim}
A: matrix(
 [3,1,-1],
 [2,2,-1],
 [2,2,0]
);
\end{verbatim}}
\end{minipage}
%%% OUTPUT:
\definecolor{labelcolor}{RGB}{100,0,0}
\begin{math}\displaystyle
\parbox{8ex}{\color{labelcolor}(\%o1) }
\begin{pmatrix}3 & 1 & -1\cr 2 & 2 & -1\cr 2 & 2 & 0\end{pmatrix}
\end{math}
%%%%%%%%%%%%%%%

Y posteriormente usamos el comando \texttt{nondiagonalizable}, siempre calculando primero los vectores propios de la
matriz.


\noindent
%%%%%%%%%%%%%%%
%%% INPUT:
\begin{minipage}{8ex}{\color{red}\bf
\begin{verbatim}
(%i4)
\end{verbatim}}
\end{minipage}
\begin{minipage}{\textwidth}{\color{blue}
\begin{verbatim}
eigenvectors(A);
\end{verbatim}}
\end{minipage}
%%% OUTPUT:
\definecolor{labelcolor}{RGB}{100,0,0}
\begin{math}\displaystyle
\parbox{8ex}{\color{labelcolor}(\%o4) }
[[[1,2],[1,2]],[[[1,0,2]],[[1,1,2]]]]
\end{math}
%%%%%%%%%%%%%%%

\noindent
%%%%%%%%%%%%%%%
%%% INPUT:
\begin{minipage}{8ex}{\color{red}\bf
\begin{verbatim}
(%i5)
\end{verbatim}}
\end{minipage}
\begin{minipage}{\textwidth}{\color{blue}
\begin{verbatim}
nondiagonalizable;
\end{verbatim}}
\end{minipage}
%%% OUTPUT:
\definecolor{labelcolor}{RGB}{100,0,0}
\begin{math}\displaystyle
\parbox{8ex}{\color{labelcolor}(\%o5) }
true
\end{math}
%%%%%%%%%%%%%%%

Si la respuesta es \texttt{true}, esto quiere decir que en efecto, tal matriz no es diagonalizable. En otro caso,
obtenendremo \texttt{false}.
\end{solucion}

¿Porqué decimos que una matriz es diagonalizable? Consideremos una transformación lineal $T:\R^{2}\to
\R^{2},$ y una base $$
F=\basis{v_{1}, v_{2}}
$$
de valores propios. Como $T(v_{1})=\lam_{1} v_{1}$, $T(v_{a})=\lam_{2} v_{2}$ y en terminos de esta base
$$
v_{1}=\bm{1\\0}_{F}, v_{2}=\bm{0\\1}_{F},
$$
entonces
$$
T\left( \bm{1\\0}_{F} \right)=\lam_{1}\bm{1\\0}_{F}=\bm{\lam_{1}\\0}_{F}
$$
y de manera similar
$$
T\left( \bm{0\\1}_{F} \right)=\lam_{2}\bm{0\\1}_{F}=\bm{0\\\lam_{2}}_{F}.
$$
Entonces, la representación matricial $D$ de la transformación $T$ en la base $F$ estará formada por los dos vectores
columna, que resultan de aplicar la transformación a cada elemento de la base, es decir,
$$
D=\bm{\lam_{1} & 0 \\ 0 & \lam_{2}}.
$$

Este mismo razonamiento, lo podemos aplicar a cualquier tranformación lineal $T:\R^{n} \to \R^{n},$ si podemos obtener
una base de vectores propios para su representación matricial $A$ (en la base estandar o de hecho, en cualquier otra
base), es decir, si $A$ es diagonalizable.

% \begin{figure}
%  \centering
%  \includegraphics[height=5cm,keepaspectratio=true]{./Sketch94203650.png}
%  % Sketch94203650.png: 720x1024 pixel, 72dpi, 25.40x36.12 cm, bb=0 0 720 1024
%  \caption{Diagonalización}
%  \label{fig:cd}
% \end{figure}


En este caso, ¿cómo podemos relacionar las representaciones matriciales de $T:\R^{n} \to \R^{n},$ en la base estandar y
en una base de vectores propios? Denotemos por $A$ a la primera y por $D$ a la segunda, mientras que por
$V=(\R^{n},E)$ al espacio vectorial $\R^{n}$ en la base $E$ estandar, mientra que $V'=(\R^{n}, F)$ en la de valores
propios. Considere el diagrama \ref{fig:cd}, donde $P$ denota la matriz cambio de base $P_{F,E}$. Es claro que
$$
AP=PD,
$$
y por tanto, multiplicando por $P^{-1}=P_{E,F}$ por la izquierda en ambos lados de la ecuación, $$D=P^{-1}AP.$$

En este caso, decimos que $D$ es una matriz diagonal \emph{semejante} a $A.$
Para un repaso de cambios de base, consulte la sección \ref{sec:coordenadas}.


\begin{problema}
 Determine si la matriz $$A=
\begin{bmatrix}
 3 & 2 & 4 \\
 2 & 0 & 2 \\
 4 & 2 & 3
\end{bmatrix}
 $$
 es diagonalizable y encuentre una matriz diagonal semejante.
\end{problema}

\begin{solucion}
 Para encontrar los vectores propios, podemos proceder como en la sección \ref{sec:eigenvectors}. Para hacer más
eficientes los cálculos, usaremos \texttt{WxMaxima}.
 Primero, introducimos la matriz:


\noindent
%%%%%%%%%%%%%%%
%%% INPUT:
\begin{minipage}{8ex}{\color{red}\bf
\begin{verbatim}
(%i1)
\end{verbatim}}
\end{minipage}
\begin{minipage}{\textwidth}{\color{blue}
\begin{verbatim}
A: matrix(
 [3,2,4],
 [2,0,2],
 [4,2,3]
);
\end{verbatim}}
\end{minipage}
%%% OUTPUT:
\definecolor{labelcolor}{RGB}{100,0,0}
\begin{math}\displaystyle
\parbox{8ex}{\color{labelcolor}(\%o1) }
\begin{pmatrix}3 & 2 & 4\cr 2 & 0 & 2\cr 4 & 2 & 3\end{pmatrix}
\end{math}
%%%%%%%%%%%%%%%

Después, encontramos los valores propios:


\noindent
%%%%%%%%%%%%%%%
%%% INPUT:
\begin{minipage}{8ex}{\color{red}\bf
\begin{verbatim}
(%i2)
\end{verbatim}}
\end{minipage}
\begin{minipage}{\textwidth}{\color{blue}
\begin{verbatim}
eigenvectors(A);
\end{verbatim}}
\end{minipage}
%%% OUTPUT:
\definecolor{labelcolor}{RGB}{100,0,0}
\begin{math}\displaystyle
\parbox{8ex}{\color{labelcolor}(\%o2) }
[[[8,-1],[1,2]],[[[1,\frac{1}{2},1]],[[1,0,-1],[0,1,-\frac{1}{2}]]]]
\end{math}
%%%%%%%%%%%%%%%

La salida de la última instrucción quiere decir que $\lam=8$ es un vector propio, de multiplicidad 1 con vector propio$$
v_{1}=\bm{1 \\ 1/2 \\ 1},
$$
mientras que $\lam_{1}=-1$ es un vector propio, de multiplicidad 2 y por tanto, los siguientes dos vectores
$$
v_{2}=\bm{1\\0\\-1}, v_{3}=\bm{0\\1\\-1/2}
$$
son vectores
propios, linealmente independientes asociados a $\lam_{2}=-1.$

Por lo tanto,
$$
P=\bm{1&1&0\\ 1/2 & 0 & 1 \\ 1 & -1 &-1/2 }.
$$

Introducimos esta matriz en \texttt{WxMaxima} y calculamos su inversa, a la que denotamos por $Q.$


\noindent
%%%%%%%%%%%%%%%
%%% INPUT:
\begin{minipage}{8ex}{\color{red}\bf
\begin{verbatim}
(%i5)
\end{verbatim}}
\end{minipage}
\begin{minipage}{\textwidth}{\color{blue}
\begin{verbatim}
P: matrix(
 [1,1,0],
 [1/2,0,1],
 [1,-1,-1/2]
);
\end{verbatim}}
\end{minipage}
%%% OUTPUT:
\definecolor{labelcolor}{RGB}{100,0,0}
\begin{math}\displaystyle
\parbox{8ex}{\color{labelcolor}(\%o5) }
\begin{pmatrix}1 & 1 & 0\cr \frac{1}{2} & 0 & 1\cr 1 & -1 & -\frac{1}{2}\end{pmatrix}
\end{math}
%%%%%%%%%%%%%%%


\noindent
%%%%%%%%%%%%%%%
%%% INPUT:
\begin{minipage}{8ex}{\color{red}\bf
\begin{verbatim}
(%i6)
\end{verbatim}}
\end{minipage}
\begin{minipage}{\textwidth}{\color{blue}
\begin{verbatim}
Q:invert(P);
\end{verbatim}}
\end{minipage}
%%% OUTPUT:
\definecolor{labelcolor}{RGB}{100,0,0}
\begin{math}\displaystyle
\parbox{8ex}{\color{labelcolor}(\%o6) }
\begin{pmatrix}\frac{4}{9} & \frac{2}{9} & \frac{4}{9}\cr \frac{5}{9} & -\frac{2}{9} & -\frac{4}{9}\cr -\frac{2}{9} &
\frac{8}{9} & -\frac{2}{9}\end{pmatrix}
\end{math}
%%%%%%%%%%%%%%%

Finalmente, realizamos el calculo $P^{-1}AP$

\noindent
%%%%%%%%%%%%%%%
%%% INPUT:
\begin{minipage}{8ex}{\color{red}\bf
\begin{verbatim}
(%i7)
\end{verbatim}}
\end{minipage}
\begin{minipage}{\textwidth}{\color{blue}
\begin{verbatim}
Q.A.P;
\end{verbatim}}
\end{minipage}
%%% OUTPUT:
\definecolor{labelcolor}{RGB}{100,0,0}
\begin{math}\displaystyle
\parbox{8ex}{\color{labelcolor}(\%o7) }
\begin{pmatrix}8 & 0 & 0\cr 0 & -1 & 0\cr 0 & 0 & -1\end{pmatrix}
\end{math}
%%%%%%%%%%%%%%%

para verificar que, en efecto, la matriz resultante es diagonal, y en su diagonal estan ordenados los valores propios
de $A.$
\end{solucion}

\subsection*{Ejemplos}

\begin{problema}
 Determine si cada matriz $A$ en el ejercicio  \ref{exe:diagonal} son diagonalizables, y en caso de serlo, encuentre
 \begin{enumerate}
  \item Una base $F$ de vectores propios de A;
  \item la matriz $P=P_{F,E}$ cambio de base, donde  $E$ es la base estandar del respectivo espacio vectorial;
  \item la matriz diagonal $D$ semejante a $A,$ usando la matriz cambio de base $P.$
 \end{enumerate} 

\end{problema}


