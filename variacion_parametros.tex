 \section{Variaci\'on de parametros}


La t\'ecnica de variaci\'on de parametros es otra forma de encontrar una soluci\'on particular de la ecuaci\'on diferencial:
\[
	\label{bron:12.1}
	L(y)=\phi(x)
\]
una vez que conocemos la soluci\'on de $L(y)=0.$



Recordemos que la soluci\'on de $L(y)=0$ está dada por
\[
	\label{bron:12.2}
	y_{h}(x)=c_{1}y_{1}(x)+...+c_{n}y_{n}(x)
\]



\subsection{El M\'etodo}


Una soluci\'on particular de $L(y)=\phi(x)$ tiene la forma
\[
	\label{bron:12.3}
	y_{p}(x)=\nu_{1}(x)\cdot y_{1}(x)+...+\nu_{n}(x) \cdot y_{n}(x)
\]
donde $y_{i}(x), \ i=1,2,...,n$ están dadas por \eqref{bron:12.2} y $\nu_{i}(x), \ i=1,2,...,n$ son funciones por determinar.



Para esto, primero resolvemos el siguiente sistema
\[
	\label{bron:12.4}
	\begin{cases}
		\nu_{1}'y_{1}+...+\nu_{n}'y_{n}&=0\\
		\nu_{1}'y'_{1}+...+\nu_{n}'y'_{n}&=0\\
		\ldots & \\
		\nu_{1}'y^{(n-1)}_{1}+...+\nu_{n}'y^{(n-1)}_{n}&=\phi(x)
	\end{cases}
\]
Posteriormente, integramos cada $\nu_{i}'(x)$ para obtener $\nu(x).$



Como $y_{1}(x),...,y_{n}(x)$ son soluciones linealmente independientes de la misma ecuaci\'on $L(y)=0,$ su wronskiano nunca se anula (teorema \ref{bron:thm:8.3}), y esto significa que el sistema \eqref{bron:12.4} tiene determinante siempre diferente de cero, y por tanto se puede resolver de manera única para $v'_{1}(x),...,v'_{n}(x).$



\begin{observacion}
	El m\'etodo de variaci\'on de parametros puede ser aplicado a todas las ecuaciones diferenciales lineales, y por tanto tiene un mayor alcance que el m\'etodo de coeficientes indeterminados.
	
	Sin embargo, si ambos m\'etodos son aplicables, es preferible el de coeficientes indeterminados por ser más eficiente.
	
	Además, en algunos casos es imposible obtener una forma cerrada de la integral de $v'_{i}(x),$ y otros m\'etodos deben ser aplicados.
\end{observacion}



\subsection{Ejemplos}


\begin{problema}
	\label{bron:exmp:12.1}
	Resuelva $y'''+y'=\sec(x)$
\end{problema}




\begin{problema}
	\label{bron:12.2}
	Resuelva $y'''-3y''+2y'=\dfrac{e^{x}}{1+e^{-x}}$
\end{problema}




\begin{problema}
	\label{bron:12.3}
	Resuelva $y''-2y'+y=\dfrac{e^x}{x}$
\end{problema}



\subsection{Ejemplos de valor inicial}


\label{bron:13.1}
\begin{problema}
	Resuelva
	$$
	y''-y'-2y=4x^{2}, \ y(0)=1, y'(0)=4.
	$$
\end{problema}




\begin{problema}
	\label{bron:13.4}
	Resuelva
	$$\begin{cases}
		y'''-6y''+11y'-6y=0, \\
		y(\pi)=0, y'(\pi)=0, y''(\pi)=1.
	\end{cases}
	$$
\end{problema}




