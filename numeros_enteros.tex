%101
\section{Los números enteros}
	
	En esta sección analizaremos algunos conjuntos numéricos. 
	Los números naturales son el conjunto de números
	$$
	\N=\set{0,1,2,3,...},
	$$	
	mientras que los números enteros son el conjunto de números
	$$
	\Z=\set{0, \pm1, \pm2,...}	$$	


\subsection{Máximo Común Divisor}
	
\begin{defn}
	Diremos que un entero $n$ divide a otro entero $c\in \Z$ si existe un tercer entero $p\in \Z$ tal que 
	$$c=n\cdot p.$$
\end{defn}

\begin{defn}
	\label{mcd} 
	Diremos que el entero $d$ es el \emph{máximo común divisor} de dos enteros $a,b$ o $\texttt{mcd}(a,b)$  si
	\begin{itemize}
		\item $d$ divide tanto a $a$ como $b$ y;
		\item $d$ es el número entero más grande con esta propiedad.
	\end{itemize}
\end{defn}

	\begin{problema}
		\label{exmp:mcd}
		Encontrar $\texttt{mcd}(6,15).$
	\end{problema}
		
	\begin{sol}
		Los divisores de $6$ son $\pm1, \pm2, \pm3, \pm6,$ mientras que los de $15$ son $\pm1, \pm3, \pm5, \pm15.$
		 		
		Entonces, los divisores en común de $6$ y $15$ son $\pm1,\pm3.$ El más grande de todos estos es 
		$d=3$ y por tanto es $$ \texttt{mcd}(6,15)=3. $$
	\end{sol}
	
 	Aunque este m\'etodo para encontrar el \texttt{mcd} es útil cuando hay pocos divisores, puede resultar abrumador si  	ambos números tienes una gran cantidad de divisores. 
   
	\begin{prop}[Teorema del Residuo]
		Dados dos números enteros positivos $a,b,$ existen otro par de enteros $q, r\geq 0$ tales que
		\begin{align}
			\label{cociente}
			a=b\cdot q+r\\
			\label{residuo}
			r<b.
		\end{align}
		A $q$ se le llama cociente, mientras que a $r$ se le llama residuo.
	\end{prop}
	\begin{proof}
		V. \cite[sección 7.2, teorema1]{cardenas1973algebra}
	\end{proof}

	\begin{problema}
		Si $a=7,b=2$, entonces el cociente es $q=3$ y el residuo es $r=1,$ porque
		$$\begin{cases}
			7=2\cdot 3+1\\ {r=1}<{b=2}.     
		\end{cases}
		$$
	\end{problema}

		Observe que tambien podr\'iamos tomar $q=1, r=5$ y escribir $$7=2\cdot 1+5,$$ pero como ${5}>{2},$ entonces $r=5$ no 
		satisface la condici\'on del residuo \eqref{residuo}, porque ${r=5}\geq {b=2}.$

	\begin{alg}[Algoritmo Euclidiano]
		Sean $a,b\in \Z$ dos números enteros positivos. Consideremos la siguiente sucesi\'on de operaciones, en la que iteramos el teorema del residuo:
		\begin{align*}
			a&=b\cdot q_{0}+r_{0} \\
			b&=r_{0}\cdot q_{1}+r_{1}\\
			r_{0}&=r_{1}\cdot q_{2}+r_{2}\\
			&...\\
			r_{N-3}&=r_{n-2}\cdot q_{N-1}+r_{N-1}\\
			r_{N-2}&=r_{N-1}\cdot q_{N}+0.
		\end{align*}
		Entonces el último cociente $r_{N-1}$ es el \texttt{mcd} de $a$ y $b.$
	\end{alg}
	
	\begin{proof}
		V. \cite[sección 7.4, prop. 1]{cardenas1973algebra}
	\end{proof}
	
	Como en el ejemplo \ref{exmp:mcd}, tenemos que 
	\begin{align*}
		15&=6\cdot 2+3\\ 
		6&=3\cdot 2+0, 
	\end{align*}
	Entonces $r=3$ es igual a $\texttt{mcd}(15,6)$.



\subsection{M\'inimo Común Múltiplo}


	\begin{defn}
		\label{mcm} 
		Diremos que el entero positivo $m\in \Z$ es el \emph{m\'inimo común multiplo} o $\texttt{mcm}$ de dos enteros positivos $a,b$ si
		\begin{itemize}
			\item $m$ es múltiplo tanto de de $a\in \Z$ como $b\in \Z$ y;
			\item $d$ es el número entero positivo más pequeño con esta propiedad.
		\end{itemize}		
	\end{defn}
	
	\begin{prop}
		\label{prop:mcm}
		Si $a,b$ son dos enteros positivos, entonces
		$$
		\texttt{mcm}(a,b)\texttt{mcd}(a,b)= a\cdot b
		$$
	\end{prop}
	\begin{proof}
		V. \cite[sección 7.5]{cardenas1973algebra}, ejercicios del 10 al 12. 
	\end{proof}
	\begin{problema}
		\label{exmp:mcm}
		Encontrar el $\texttt{mcm}$ de $a=6$ y $b=15.$
	\end{problema}
	\begin{sol}
		Como vimos anteriormente, $ \mcd(6,15)=3 $. Entonces, 
		\[ \mcm(6,15)=\dfrac{6\cdot15}{3}= 30 \]
	\end{sol}

