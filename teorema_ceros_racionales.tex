
%302
\section{Teorema de los ceros racionales}


	Decimos que $c$ es un \emph{cero racional} del polinomio $p(x)$ si $p(c)=0$ y $c$ es un número racional, es decir, una fracci\'on.



	\begin{rem}
		No todo cero de un polinomio es racional. Por ejemplo, los ceros del polinomio $p(x)=x^{2}-2$ son $c=\pm\sqrt{2},$ y desde los tiempos de Pitágoras es sabido que \href{https://www.youtube.com/watch?v=gVkB3XuK6MU}{las ra\'ices de números primos no son números racionales.}
	\end{rem}
	




	\begin{thm}[Teorema de los ceros racionales, caso particular]
		Si el polinomio {\color{blue}$p(x)=x^{n}+a_{n-1}x^{n-1}+...+a_{1}x+a_{0}$} tiene {\color{green}coeficientes enteros}, entonces \emph{todo cero racional es divisor de t\'ermino constante $a_{0}.$}
	\end{thm}
	



	\begin{problema}
		Hallar los ceros racionales de 
		$$
		p(x)=x^{3}-3x+2.
		$$
	\end{problema}
	



	\begin{thm}[Teorema de los ceros racionales, caso particular]
		Si el polinomio {\color{blue}$$p(x)=a_{n}x^{n}+a_{n-1}x^{n-1}+...+a_{1}x+a_{0}, \; a_{n}\neq 0$$} tiene {\color{green}coeficientes enteros}, entonces \emph{todo cero racional de la forma $\displaystyle\dfrac{p}{q}$ d\'onde $p$ es divisor de coeficiente constante $a_{0}$ y $q$ es divisor de coeficiente l\'ider $a_{n}$.} 
	\end{thm}



	\begin{problema}
		Encuentre los ceros racionales del polinomio $$p(x)=2x^{3}+x^{2}-13x+6.$$
	\end{problema}
	


% 
% 
% 	\begin{defn}
% 		Un número entero $p$ es llamado \emph{primo} si tiene exactamente cuatro divisores, que en ese caso serán, $\pm 1, \pm p.$  
% 	\end{defn}
% 
% 
% 
% 	\begin{problema}
% 		Encuentre todos los número primos menores a 40.
% 		\begin{hint}
% 			Utilice la \emph{cripta de Erat\'ostenes.}
% 		\end{hint}
% 		
% 	\end{problema}
% 	
% 
% 
% 
% 	Como los números \emph{$\pm 1$} reciben un nombre especial, y son llamados \emph{unidades.}
% 
% 
% 
% 
% 	\begin{thm}[Factorizaci\'on prima]
% 		Para cada número entero $n\neq \pm 1,$ existe una unidad $u$, una lista de números primos $$\set{ p_{1},...,p_{m}}, m>0$$ y una lista de potencias $\set{R_{1},...,R_{m}}$ con cada $R_{i}>0$, para $ i=1,...,m$ tal que 
% 		$$
% 		n=u\cdot p_{1}^{R_{1}}\cdots p_{m}^{R_{m}}.
% 		$$
% 		Más aun, la elecci\'on de la unidad y de las listas es única. 
% 	\end{thm}
% 	
% 
% 
% 
% 	\begin{problema}
% 		Factorice los siguientes números:
% 		\begin{multicols}{2}
% 			\begin{enumerate}
% 				\item $7840$  $=2^{5}\cdot5\cdot7^{2}$
% 				\item $4860$  $=2^{2}\cdot 3^{5} \cdot 5$
% 				\item $8624$  $=2^{4}\cdot 7^{2}\cdot 11$
% 				\item $2940$  
% 				$=2^{2}\cdot 3 \cdot 5 \cdot 7^{2}$
% 				\item $4050$ 
% 				$= 2\cdot 3^{4} \cdot 5^{2}$
% 				\item $3234$ 
% 				$=2 \cdot 3 \cdot 7^{2} \cdot 11$
% 				\item $8575$ 
% 				$=5^{2} \cdot 7^{3}$
% 				\item $1512$ 
% 				$=2^{3}\cdot 3^{3} \cdot7$
% 				\item $5850$ 
% 				$=2\cdot 3^{2} \cdot 5^{2} \cdot 13$
% 				\item $6912$
% 				$=2^{8}\cdot 3^{3}$
% 			\end{enumerate}
% 			
% 			
% 		\end{multicols}
% 		
% 		
% 	\end{problema}
% 	
% 
% 
% 
% 	\begin{problema}
% 		Factorice los siguientes números enteros:
% 		\begin{multicols}{2}
% 			\begin{enumerate}
% 				\item $4116$
% 				\item $3150$
% 				\item $6600$
% 				\item $4212$
% 				\item $1920$
% 				\item $640$
% 				\item $3696$
% 				\item $4455$
% 				\item $9072$
% 				\item $6174$
% 			\end{enumerate}
% 			
% 		\end{multicols}
% 		
% 	\end{problema}
% 	
% 
% 
% 
% 	\begin{alg}[Como encontrar todos los divisores de un número entero]
% 		\begin{enumerate}
% 			\item Factorice el número entero
% 			$n=p_{1}^{R_{1}}\cdots p_{m}^{R_{m}}$
% 			\item Enliste cada posible $m-$tupla
% 			$\left( r_{1},...,r^{m} \right)$
% 			con $0\leq r_{1}\leq R_{1},...,0\leq r_{m}\leq R_{m}$
% 			\item Enliste cada posible número entero de la forma $$\pm p_{1}^{r_{1}}\cdots p_{m}^{r_{m}},$$ para cada elemento $\left( r_{1},...,r_{m} \right)$ de la lista anterior.
% 		\end{enumerate}
% 	\end{alg}
% 	
% 
% 
% 
% 	\begin{rem}
% 		Con la notaci\'on anterior, el número exacto de divisores positivos de $n$ será 
% 		$$
% 		(R_{1}+1)\times \cdots \times(R_{m}+1).
% 		$$
% 	\end{rem}
% 	
% 
% 
% 
% 	\begin{problema}
% 		Encuentre todos los divisores positivos de 
% 		\begin{multicols}{2}
% 			\begin{enumerate}
% 				\item $288$
% 				\item $540$
% 				\item $600$
% 				\item $567$
% 				\item $896$
% 				\item $675$
% 				\item $504$
% 				\item $640$
% 				\item $810$
% 				\item $672$
% 			\end{enumerate}
% 			
% 		\end{multicols}
% 		
% 	\end{problema}
% 
% 
% 
% 	\begin{problema}
% 		Encuentre todos los divisores positivos de
% 		\begin{multicols}{2}
% 			\begin{enumerate}
% 				\item $324$
% 				\item $192$
% 				\item $840$
% 				\item $720$
% 				\item $980$
% 				\item $336$
% 				\item $420$
% 				\item $300$
% 				\item $972$
% 				\item $486$
% 			\end{enumerate}
% 			
% 		\end{multicols}
% 		
% 	\end{problema}
% 	
% 
% 
% 
% 	\begin{problema} Encuentre los ceros racionales del polinomio
% 		\begin{multicols}{2} 
% 			\begin{enumerate}
% 				\item $x^{3}+3x^{2}-4$
% 				\item $x^{3}-x^{2}-8x+12$
% 				\item $x^{4}-5x^{2}+4$
% 				\item $x^{4}-x^{3}-5x^{2}+3x+6$
% 				\item $4x^{3}-7x+3$
% 				\item $6x^{4}-7x^{3}-12x^{2}+3x+2$
% 				\item $2x^{6}-3x^{5}-13x^{4}+29x^{3}-27x^{2}+32x-12$
% 			\end{enumerate}
% 			
% 		\end{multicols}
% 	\end{problema}
% 
