\section{Límites y continuidad}

\subsection{Límites}
{Definición}
 Diremos que la función tiene un \emph{límite $L$ cuando $x$ aproxima $a$} si para cada $\epsilon >0$, podemos encontrar $\delta >0$ tal que 
 \begin{align*}
  \abs{f(x)-l} < \epsilon
  \end{align*}
siempre que 
\begin{align*}
 0 < \abs{x-a} < \delta.
 \end{align*}

%%%%%%%%%%%%%%%%%%%%%
{}
  En ese caso, escribimos
  \begin{align*}
   \lim_{x \to a} f(x) = L
   \end{align*}

%%%%%%%%%%%%%%%%%%%%%
{}
  \begin{problema}
   Calcula los siguientes límites, trazando las gráficas correspondientes:
   \begin{enumerate}[(i)]
     %NUEVO ITEM
     \item $\lim_{x\to 1}\left( x^{2}-4x+8 \right)$      
     \item $\lim_{x\to 2}\dfrac{x^{2}-4}{x-2}$      
     \item $\lim_{x\to 0}\dfrac{\sin(x)}{x}$ 
\end{enumerate}
  \end{problema}


%%%%%%%%%%%%%%%%%%%%%
{Propiedades de límites}
  Si $\lim_{x \to a}f_{1}(x)= L_{1}$ y $\lim_{x \to a }f_{2}(x)= L_{2}$, entonces
  \begin{enumerate}[(i)]
    %NUEVO ITEM
    \item $\lim_{x\to a}\left( f_{1}(x)\pm f_{2}(x) \right) = L_{1}\pm L_{2}$     
    \item $\lim_{x\to a}\left( f_{1}(x) f_{2}(x)\right)= L_{1}L_{2}$.    
    \item 
    $\lim_{x \to a}\dfrac{f_{1}(x)}{f_{2}(x)} = \dfrac{L_{1}}{L_{2}}$ 
    \emph{siempre y cuando $L_{2}\neq 0$}.
\end{enumerate}

%%%%%%%%%%%%%%%%%%%%%
\subsection{Continuidad}
{}
  Diremos que una función es \emph{continua en el punto $x=a$} si 
  \begin{align*}
   \lim_{x\to a} f(x)=f(a)
   \end{align*}

  \begin{problema}
   $f(x)=x^{2}-4x+8$ es continua en $x=1$.
  \end{problema}


  \begin{problema}
   Sin embargo, 
   \begin{align*}
   f(x)=
    \begin{cases}
\dfrac{x^2-4}{x-2}& x\neq 2\\
6 & x= 2
\end{cases}
    \end{align*}
    es \emph{discontinua} en $x=2$  y decimos que $x=2$ es una \emph{discontinuidad}.
  \end{problema}


%%%%%%%%%%%%%%%%%%%%%
{}
  Si $f(x)$ es continua en cada punto del intervalo $x_{1}<x<x_{2}$, entonces diremos que es continua en dicho intervalo.

%%%%%%%%%%%%%%%%%%%%%
{}
  \begin{proposicion}
   Si $f(x)$ y $g(x)$ son continuas en un mismo intervalo, entonces también lo son
   \begin{enumerate}[(i)]
     %NUEVO ITEM
     \item $f(x)\pm g(x)$ 
     
     \item $f(x)g(x)$
     
     \item $\dfrac{f(x)}{g(x)}$ siempre que $g(x)\neq 0$ en dicho intervalo.
\end{enumerate}
  \end{proposicion}


%%%%%%%%%%%%%%%%%%%%%
\subsection{Ejemplos}
{}
  \begin{problema}
   Demostrar por definición que $\lim_{x\to 2}f(x)=4$ si
   \begin{enumerate}[(i)]
     %NUEVO ITEM
     \item $f(x)=x^{2}$ 
     
     \item $f(x)= \begin{cases}
x^{2} & x\neq 2 \\
0 & x =2
\end{cases}$
\end{enumerate}
  \end{problema}


%%%%%%%%%%%%%%%%%%%%%
{}
  \begin{problema}
  Demostrar que si $\lim_{x\to a}f_{1}(x)=l_{1}$ y $\lim_{x\to a}f_{2}(x)=l_{2}$, entonces 
  \begin{align*}
   \lim_{x\to a}\left( f_{1}(x)+f_{2}(x) \right)=l_{1}+l_{2}
   \end{align*}
  \end{problema}


%%%%%%%%%%%%%%%%%%%%%
{}
  \begin{problema}
   Mostrar que $f(x)=x^{2}$ es continua en $x=2$, pero $f(x)= \begin{cases}
x^{2} & x\neq 2 \\
0 & x =2
\end{cases}$ no lo es. 
  \end{problema}


%%%%%%%%%%%%%%%%%%%%%
