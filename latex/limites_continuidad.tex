\section{Límites y continuidad}

\subsection{Definición informal de límite}

De manera informal, el límite $ L $ de una función $ f(x) $ cuando $ x $ se aproxima a un valor $ c $ es el valor al que esperamos que se acerque $ f(x) $.

Si tal valor existe, diremos que $ f(x)\to L $ ( que se lee  $f(x)  $ tiende a $ L $) cuando $ x \to  $ (que leeremos como $ x $ tiende a $ c $).

Otra manera de escribir este resultado es
\[ L = \lim_{x\to c} f(x) ,\]
que se lee como $ L $ es el límite de $ f(x) $ cuando $ x $ tiende a $ c $.


\begin{resuelto}
	Estima los siguientes límites, trazando las gráficas correspondientes:
	\begin{enumerate}[(i)]
		%NUEVO ITEM
		\item $\lim_{x\to 1}\left( x^{2}-4x+8 \right)$
%		\item $ \lim_{x\to 2} f(x) $ donde
%		$$f(x)= \begin{cases}
%			x^{2} & x\neq 2 \\
%			0 & x =2
%		\end{cases}$$
		\item $\lim_{x\to 2}\dfrac{x^{2}-4}{x-2}$
		\item $\lim_{x\to 0}\dfrac{\sin(x)}{x}$
		\item $ \lim_{x \to 0} x^{x} $
	\end{enumerate}
	¿Qué ocurre si intentas evaluar inmediatamente en la función?
\end{resuelto}

Sin embargo, $ x $ puede tender a $ c $ en dos direcciones:
\begin{enumerate}
	\item $ x $ tiende a $ c $ por la izquierda, es decir, $ x $ es menor a $ c $ pero se incrementa poco a poco, y se acerca a $ c $ cada vez más. En ese caso, escribiremos $ x\to c^{-} .$
	\item $ x $ tiende a $ c $ por la derecha, es decir, $ x $ es mayor a $ c $ pero se dencrementa poco a poco, y se acerca a $ c $ cada vez más. En ese caso, escribiremos $ x\to c^{+} .$
\end{enumerate}

\begin{observacion}
	A estos límites se le conoce como límites laterales. Cuando ambos límites laterales coinciden, entonces el valor común es \textbf{el límite}.
\end{observacion}

\begin{observacion}
	Nota los súper-índice en la $ c^{-}, c^{+} $. Es muy fácil pasarlos por alto al momento de calcular los límites laterales, o confundirlas con un exponente.
\end{observacion}

\begin{resuelto}
	Calcula los límites laterales de las siguientes funciones en cuando $ x $ tiende a cero y determina si existe el límite.
	\begin{enumerate}
		\item $ f(x) = |x| $
		\item $ g(x) = \operatorname{RELU}(x) $
		\item $ H(x) $, la función de Heaviside.
	\end{enumerate}
\end{resuelto}

\subsection{Definición formal de límite}

\begin{definicion}

	Diremos que la función tiene un \emph{límite $L$ cuando $x$ aproxima $a$} si para cada $\epsilon >0$, podemos encontrar $\delta >0$ tal que
	\begin{align}
		\abs{f(x)-l} < \epsilon
	\end{align}
	siempre que
	\begin{align}
		0 < \abs{x-a} < \delta.
	\end{align}

 En ese caso, escribimos
\begin{align}
	\lim_{x \to a} f(x) = L
\end{align}
\end{definicion}


\begin{resuelto}
	Demostrar por definición que $$\lim_{x\to 2}x^2=4$$.
\end{resuelto}


\begin{proposicion}

	Si $\lim_{x \to a}f_{1}(x)= L_{1}$ y $\lim_{x \to a }f_{2}(x)= L_{2}$, entonces
	\begin{enumerate}[(i)]
		%NUEVO ITEM
		\item $\lim_{x\to a}\left( f_{1}(x)\pm f_{2}(x) \right) = L_{1}\pm L_{2}$
		\item $\lim_{x\to a}\left( f_{1}(x) f_{2}(x)\right)= L_{1}L_{2}$.
		\item
		$\lim_{x \to a}\dfrac{f_{1}(x)}{f_{2}(x)} = \dfrac{L_{1}}{L_{2}}$
		\emph{siempre y cuando $L_{2}\neq 0$}.
	\end{enumerate}
\end{proposicion}


\begin{resuelto}
	Demostrar que si $\lim_{x\to a}f_{1}(x)=l_{1}$ y $\lim_{x\to a}f_{2}(x)=l_{2}$, entonces
	\begin{align}
		\lim_{x\to a}\left( f_{1}(x)+f_{2}(x) \right)=l_{1}+l_{2}
	\end{align}
\end{resuelto}


\subsection{Continuidad}

  Diremos que una función es \emph{continua en el punto $x=a$} si
  \begin{align}
   \lim_{x\to a} f(x)=f(a)
   \end{align}

  \begin{resuelto}
   Verifica que $f(x)=x^{2}-4x+8$ es continua en $x=1$, pero que
   \begin{align}
   f(x)=
    \begin{cases}
\dfrac{x^2-4}{x-2}& x\neq 2\\
0 & x= 2
\end{cases}
    \end{align}
    es \emph{discontinua} en $x=2$.    En ese caso, decimos que $x=2$ es una \emph{discontinuidad}.
  \end{resuelto}

\begin{resuelto}

 \begin{enumerate}
 	\item Considera la función
 	\[ f(x) = \dfrac{\sin(x)}{x}, x\neq 0 .\]

 	¿Qué valor debería tener $ f(0) $ para que esta función sea continua? \href{https://mathworld.wolfram.com/SincFunction.html}{Esta función, definida así en toda la recta real se le llama $ sinc(x) $. }
 	\item ¿Es posible hacer algo similar con
 	$ g(x) = x^x $, es decir, hacerla continua en $ x=0 $?
 \end{enumerate}
\end{resuelto}

\begin{definicion}
	Si $f(x)$ es continua en cada punto del intervalo $x_{1}<x<x_{2}$, entonces diremos que es continua en dicho intervalo.
\end{definicion}

  \begin{proposicion}
   Si $f(x)$ y $g(x)$ son continuas en un mismo intervalo, entonces también lo son
   \begin{enumerate}[(i)]
     %NUEVO ITEM
     \item $f(x)\pm g(x)$

     \item $f(x)g(x)$

     \item $\dfrac{f(x)}{g(x)}$ siempre que $g(x)\neq 0$ en dicho intervalo.
\end{enumerate}
  \end{proposicion}

\begin{resuelto}
	Para cada una de las siguientes funciones, determina el intervalo maximal en el que son continuas:
	\begin{enumerate}
		\item polinomios
		\item exponenciales
		\item logaritmos
		\item trigonométricas
		\item inversas trigonométricas
	\end{enumerate}
\end{resuelto}