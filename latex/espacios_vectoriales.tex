\section{Definici\'on y ejemplos}



Hasta ahora hemos considerado a los vectores como elementos de un espacio $$\R^{n}=\set{(x_{1},...,x_{n}|x_{k}\in\R)},$$ por ejemplo vectores en el plano $\R^{2}=\set{(x,y)|x,y\in \R}$ o en el espacio $\R^{3}=\set{(x,y,z)|x,y,z\in \R}.$ En cada caso, ten\'iamos una suma entre vectores y una multiplicaci\'on por \emph{escalares,} es decir, número reales.

\begin{problema}
 Si ${u}=(u_{1},u_{2})$ y ${v}=(v_{1},v_{2})$ son vectores en $\R^{2},$ y $\a \in \R,$ entonces $$(u_{1},u_{2})+(v_{1},v_{2})=(u_{1}+v_{1},u_{2}+v_{2}),$$ mientras que $$\a(u_{1},u_{2})=(\a u_{1},\a u_{2}).$$


 En este caso, la suma tiene las siguientes propiedades:
 \begin{enumerate}
  \item(Cerradura)  ${u}+{v}=(u_{1},u_{2})+(v_{1},v_{2})=(u_{1}+v_{1},u_{2}+v_{2}),$ es un vector en $\R^{2},$
  \item(Asociatividad) Si $w=(w_{1},w_{2})$ es otro vector en $\R^{2},$ entonces
  \begin{align*}
(u+v)+w&=\left( (u_{1},u_{2})+(v_{1},v_{2}) \right) + (w_{1},w_{2}) \\
&=(u_{1},u_{2}) + \left( (v_{1},v_{2}) + (w_{1},w_{2}) \right) \\
&=u+(v+w),
  \end{align*}
\item(Conmutatividad)
$
u+v=(u_{1}+v_{1},u_{2}+v_{2})=(v_{1}+u_{1},v_{2}+u_{2})=v+u
$
\item(Existencia de un elemento neutro)
$
u+\vec{0}=(u_{1},u_{2})+(0,0)=(u_{1},u_{2})=u,
$
y de la misma forma $\vec{0}+u=u.$
\item(Inversos aditivos) Para $u=(u_{1},u_{2}),$ definimos $$-u=(-u_{1},-u_{2}),$$ y este elemento satisface que
$$
u+(-u)=(-u)+u=0.
$$
 \end{enumerate}


 La multiplicaci\'on por escalares satisface las siguientes  propiedades
 \begin{enumerate}
 \item $\a u$ es de nuevo un vector en $\R^{2},$
  \item $1u=1(u_{1},u_{2})=(1\cdot u_{1},1\cdot u_{2})=u,$
  \item $(\a\b) u=\a(\b u_{1}, \b u_{2})=\a(\b u).$
 \end{enumerate}

 Finalmente, la suma de vectores y la multiplicaci\'on por escalares estan relacionadas por las siguiente leyes distributivas.
 \begin{enumerate}
  \item $(\a+\b)u=((\a+\b)u_{1},(\a+\b)u_{2})=(\a u_{1},\a u_{2})+(\b u_{1},\b u_{2})=\a u +\b u,$
  \item $\a(u+v)=\a(u_{1}+v_{1},u_{2}+v_{2})=(\a(u_{1}+v_{1}),\a(u_{2}+v_{2}))=\a(u_{1},u_{2})+\a(v_{1},v_{2})=\a u +\a v.$
 \end{enumerate}
\end{problema}

\begin{problema}[\dag]
 Verificar que las mismas propiedades se cumplen para $\R^{3},$ usando la suma de vectores y multiplicaci\'on por
escalares conocida.
\end{problema}

Estas propiedades se cumplen para muchos y muy diferentes conjuntos, donde tenemos una operaci\'on suma entre sus elementos y podemos definir una multiplicaci\'on por números reales. De hecho, estos conjuntos son los objetos de estudio en el álgebra lineal.

\begin{definicion}
 Sea $V$ un conjunto, con una operaci\'on $+:V \times V \to V$ y una operaci\'on $\cdot:\R \times V \to V.$ Decimos
que $V$ es un \emph{espacio vectorial (sobre $\R$)} si para todo $u,v,w\in V$ y $\a,\b\in \R$ se cumplen las siguientes
propiedades.
 \begin{enumerate}
  \item(Cerradura)$u+v\in V,$
  \item(Asociatividad)$(u+v)+w=u+(v+w),$
  \item(Conmutatividad)$u+v=v+u,$
  \item(Elemento neutro) Existe $0\in V,$ tal que para todo $u\in V: u+0=0+u=u,$
  \item(Elementos inversos) Para todo $u\in V,$ existe $-u \in V,$ tal que $u+(-u)=(-u)+u=0.$
  \item $\a u \in V,$
 \item $1u=u,$
 \item $(\a\b)u=\a(\b u),$
\item $(\a+\b)u=\a u+\b u,$
\item $\a(u,v)=\a u + \a v.$
\end{enumerate}

A los elementos del espacio vectorial $V$ les llamamos \emph{vectores.}
\end{definicion}

\begin{observacion}
 Cuando $V$ es un espacio vectorial, con operaci\'on suma $+:V \times V \to V$ y multiplicaci\'on por escalares
$\cdot:\R \to V \to V,$ por brevedad, decimos que $(V, +, \cdot)$ es un espacio vectorial.
\end{observacion}



\subsection*{Ejemplos}

%Para resolver los ejercicios de esta secci\'on puede consultar \cite[sec. 4.2]{G} y \cite[sec. 2.1]{HK}.

\begin{problema}[\dag]
Demuestre usando las propiedades anteriores, que en cualquier espacio vectorial se cumplen las siguientes propiedades.
\begin{enumerate}
 \item $0  u=\a  0=0.$ (Note que el cero escrito a la izquierda denota el cero como número, mientras que escrito a la izquierda o solo, denota el elemento neutro  del espacio vectorial.)
 \item $-u=(-1) u.$ \emph{Sugerencia: Verifique que $u+(-1)u=0$.}
 \item Si $\a u=0,$ entonces o bien $\a=0$ o $u=0.$
 \item El elemento neutro $0$ es único.
 \item Para cada vector $u,$ su inverso aditivo $-u$ es único.
\end{enumerate}
\end{problema}

\begin{problema}
 Compruebe que los siguientes conjuntos son espacios vectoriales (reales), con las operaciones suma y multiplicaci\'on por escalar usuales.
 \begin{enumerate}
 \item $\set{0}.$
 \item $\set{(x,y)\in \R^{2}|y=mx}$ para $m\in \R$ fijo.
 \item $\set{(x,y,z)\in \R^{3}| ax+by+cz=0}$ para $a,b,c\in \R.$
 \item $\set{(x,y,z)\in \R^{3}|(a,b,c)\cdot (x,y,z)=0}$ para $a,b,c\in \R.$
  \item $\set{f|f:S\to \R},$ donde $S$ puede ser cualquier conjunto fijos.
  \item $\R^{m\times n},$ es decir, el conjunto de matrices $m\times n$ con entradas reales.
  \item  El espacio de polin\'omios con coeficientes reales.
  \item El espacio de polin\'omios con coeficientes reales de grado $\leq n,$ para $n \in \N$ fijo.
  \item $C[a,b],$ el espacio de funciones continuas en el intervalo $[a,b].$
  \item El conjunto de número reales \emph{positivos} con las operaciones $u\oplus v:= u,v$ y $\a \cdot u:=u^{\a}.$
 \end{enumerate}
\end{problema}

% \begin{problema}
%  Indique si los siguientes conjuntos son espacios vectoriales
%  \begin{enumerate}
%   \item El conjunto de matrices diagonales $2 \times 2,$ bajo la multiplicaci\'on, es decur, con la suma definida como $A\oplus B=AB.$
%   \item Los vectores en el primer cuadrante del plano.
%   \item $\set{(x,y)|y\geq 0}$
%   \item El conjunto de vectores en $\R^{3}$ de la forma $(x,x,x).$
%   \item El conjunto de matrices $2\times 2$ de la forma
%   $$
% \begin{pmatrix}
%  0 & -a\\
%  a & a\\
% \end{pmatrix}
%   $$
% \item El conjunto de funciones continuas de valores reales definidas en $[0,1]$  con $f(0)=f(1)=0.$
% \item $R^{2}$ con la suma definida como $$(x_{1},y_{1})\oplus (x_{2},y_{2})=(x_{1}+x_{2}+1, y_{1}+y_{2}+1),$$ y la multiplicaci\'on usual por escalares.
% \item El conjunto de funciones diferenciables en $[0,1]$ con las operaciones usuales.
% \item El conjunto de números reales de la forma $a+b\sqrt{2},$ donde $a,b$ son números racionales, bajo la suma usual y la multiplicaci\'on por escalares restringida a número racionales
% \item $\R^{2}$ con la suma usual y la multiplicaci\'on por escalares definida como $\a\cdot u=(\a u_{1},u_{2}).$
% \item $\R^{2}$ con las suma definida como
% $$
% u \oplus v:= u-v,
% $$
% y la multiplicaci\'on por escalares como
% $$
% \a \cdot u := -\a u.
% $$
% \item $\R^{2}$ con las suma definida como
% $$
% u \oplus v:= (u_{1}+v_{1},0),
% $$
% y la multiplicaci\'on por escalares como
% $$
% \a \cdot u := (\a u_{1},0).
% $$
% \end{enumerate}
% \end{problema}

\begin{problema}
 Considere la ecuaci\'on diferencial de segundo orden homog\'enea
 $$
y''(x)+a(x)y'+b(x)y(x)=0
 $$
donde $a,b$ son funciones continuas. Demuestre que el conjunto de soluciones de la ecuaci\'on diferencial es un espacio vectorial bajo las reglas usuales para la suma de funciones y multiplicaci\'on por escalares.
 \end{problema}





