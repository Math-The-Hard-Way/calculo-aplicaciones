\section{Vectores propios}
\label{sec:eigenvectors}


\begin{definicion}
 Si $T:\R^{n}\to \R^{n}$ es una transformación lineal y $A$ su representación matricial, en la base estandar, y $\lam$
un valor propio de $A,$ entonces cualquier vector $v\in \R^{n}$ que satisfaga la ecuación
$$
Av =\lam v
$$
se conoce como $\lam$-vector propio.

Al conjunto de $\lam$-vectores propios se le conoce como $\lam$-espacio propio y se denota por $E_{\lam}.$
\end{definicion}

\begin{observacion}
 En el caso anterior, tenemos que
 $$
E_{\lam}=\ker\left( A-\lam I \right).
 $$
\end{observacion}

\begin{problema}
 Encuentre el espacio propio asociado al valor propio $\lam_{1}=2$ de la matriz $A$ definida en (\ref{no:diagonal}).
\end{problema}

\begin{solucion}
 Si $$v=\bm{x \\ y \\ z}\in E_{\lam_{1}},$$ entonces $\left( A- 2I \right)v=0,$ es decir,
 $$
 \bm{1&1&-1\\2&0&-1\\2&2&-2}\bm{x \\ y \\ z}=\bm{0\\0\\0},
 $$
 que se reduce al sistema de ecuaciones
 $$\begin{cases}
    2x=z\\
    z=2y
   \end{cases}.
 $$
Escogiendo $z=2t,$ donde $t\in \R,$ obtenemos
$$
\bm{x \\ y \\ z}=\bm{t \\ t \\ 2t}=t\bm{1 \\ 1 \\ 2}.
$$

 Es decir $\ker(A-2I)$ esta generado por el conjunto $\set{\bm{1 \\ 1 \\ 2}}$ y al consistir de un solo vector, este es
linealmente independiente, y por tanto es una base. En resumen,
$$
\ker(A-2I)=\basis{\bm{1 \\ 1 \\ 2}}.
$$
 \end{solucion}

 \begin{problema}
  Encuentre el espacio propio asociado al valor propio $\lam_{2}=1$ de la matriz $A$ definida en (\ref{no:diagonal}).
 \end{problema}

 \begin{solucion}
  $$
\ker(A-I)=\basis{\bm{1\\0\\2}}.
  $$
 \end{solucion}


 Para comprobar nuestros resultados, podemos usar \texttt{WxMaxima}. Primero, introducimos nuestra matriz.


\noindent
%%%%%%%%%%%%%%%
%%% INPUT:
\begin{minipage}{8ex}{\color{red}\bf
\begin{verbatim}
(%i1)
\end{verbatim}}
\end{minipage}
\begin{minipage}{\textwidth}{\color{blue}
\begin{verbatim}
matrix(
 [3,1,-1],
 [2,2,-1],
 [2,2,0]
);
\end{verbatim}}
\end{minipage}
%%% OUTPUT:
\definecolor{labelcolor}{RGB}{100,0,0}
\begin{math}\displaystyle
\parbox{8ex}{\color{labelcolor}(\%o1) }
\begin{pmatrix}3 & 1 & -1\cr 2 & 2 & -1\cr 2 & 2 & 0\end{pmatrix}
\end{math}
%%%%%%%%%%%%%%%

Posteriormente, calculamos los vectores propios de la siguiente manera.


\noindent
%%%%%%%%%%%%%%%
%%% INPUT:
\begin{minipage}{8ex}{\color{red}\bf
\begin{verbatim}
(%i2)
\end{verbatim}}
\end{minipage}
\begin{minipage}{\textwidth}{\color{blue}
\begin{verbatim}
eigenvectors(%);
\end{verbatim}}
\end{minipage}
%%% OUTPUT:
\definecolor{labelcolor}{RGB}{100,0,0}
\begin{math}\displaystyle
\parbox{8ex}{\color{labelcolor}(\%o2) }
[[[1,2],[1,2]],[[[1,0,2]],[[1,1,2]]]]
\end{math}
%%%%%%%%%%%%%%%

El primer arreglo $[1,2]$ nos dice los dos valores propios, mientras que el segundo $[1,2]$ nos dice su multiplicidad
algebráica. El tercer arreglo $[1,0,2]$ es un vector propio de $\lam=1,$ mientras que el último $[1,1,2]$ es uno
asociado a $\lam=2$. Como explicamos anteriormente, cada uno de estos constituye una base de sus respectivos espacios
propios.

\subsection*{Ejemplos}

\begin{problema}
 Encuentre los espacios propios de los diferentes valores propios de las matrices dadas en el ejercicio
 \ref{exe:diagonal}.
\end{problema}

