    \section{Representación paramétrica de curvas}

\subsection{Ecuaciones paramétricas}

{}
Si las coordenadas $(x,y)$ de un punto $P$ en una curva están dadas por las funciones
\begin{align}
	\label{ecspar}
	\tag{Ec.Par.}
	x = f(t), y=g(t)
\end{align}
de una tercera variable o \emph{parámetro} \(t\), entonces \eqref{ecspar} son llamadas \emph{ecuaciones paramétricas} de la curva.


{}
\begin{resuelto}  \label{ayr:exmp:37.1.a}
	\begin{enumerate}[a]
		\item
		\begin{align}
			x=\cos(t), y = \sin^2(t)
		\end{align}
		son ecuaciones paramétricas de la parábola
		\begin{align}
			4x^2+y = 4.
		\end{align}
		\item
		\begin{align}
			x=\dfrac{1}{2}t, y = 4 - t^2
		\end{align}
		es otra parametrización de la misma curva.

	\end{enumerate}
\end{resuelto}


{}
\begin{resuelto}
	\begin{enumerate}
		\item Las ecuaciones
		\begin{align}
			x = r\cos(t), y=r \sin(t)
		\end{align}
		representan el círculo con radio en el origen.

		\item
		Las ecuaciones
		\begin{align}
			x=a+r\cos(t), y=b+r\sin(t)
		\end{align}
		representa el círculo de radio \( r \) y centro en \((a,b)\).
	\end{enumerate}

\end{resuelto}



Supongamos que la curvas está dada por \eqref{ecspar}. Entonces las primera y segunda derivadas están dadas por
\begin{align}
	\label{37.1}
	D_{x}y &= \dfrac{D_{t}y}{D_{t}x}\\
	\label{37.2}
	D_{xx}y &= \dfrac{D_{tx}y}{D_{t}x}
\end{align}



\subsection{Longitud de arco}

Si una curva está dada por \eqref{ecspar}, entonces la \emph{longitud de curva}  entre dos puntos correspondientes a los valores parámetricos \(t_{1}\) y \(t_{2}\) es
\begin{align}
	L=\int_{t_{1}}^{t_{2}}
	\sqrt{\left( D_{t}x \right)^{2}
		+\left( D_{t}y \right)^{2}} dt
\end{align}


\subsection{Ejemplos}

\begin{resuelto} %37.1
	Encuentre \(D_{x}y\), \(D_{xx}y\) para
	\begin{align}
		x=t-\sin(t), y = 1-\cos(t)
	\end{align}

\end{resuelto}



\begin{resuelto} %37.2
	Encuentre \(D_x y\) y \(D_{xx} y\) si \(x=e^t \cos(t), y = e^t\sin(t)\).
\end{resuelto}



\begin{resuelto} %37.3
	Encuentre una ecuación a la línea tangente de la curva
	\begin{align}
		x=\sqrt{t}, y=t-\dfrac{1}{\sqrt{t}}
	\end{align}
	en el punto $t=4$.
\end{resuelto}



\begin{resuelto} %37.4
	La posición de una particula que se está moviendo a lo largo de una curva está dada al tiempo $t$ por las ecuaciones paramétricas
	\begin{align}
		x=2-3\cos(t), y = 3+2 \sin(t)
	\end{align}
	donde $x$ y $y$ están medidos en pies y $t$ en segundos.


	Note que
	\begin{align}
		\dfrac{1}{9}(x-2)^2+\dfrac{1}{4} (y-3)^2 = 1
	\end{align}
	de manera que la curva es una elipse.\emph{¿Porqué?}
\end{resuelto}

{Continuación}
\begin{enumerate}
	\item Encuentre la tasa de cambio temporal de $x$ cuando $t=\dfrac{\pi}{3}$
	\item Encuentre la tasa de cambio temporal de $y$ cuando $t=\dfrac{5\pi}{3}$
	\item Encuentre la tasa de cambio temporal del ángulo de inclinación $\theta$ de la línea tangente cuando $t=\dfrac{2\pi}{3}$
\end{enumerate}


\begin{resuelto} %37.5
	Encuentre la longitud de arco  de la curva
	\begin{align}
		x=t^2, y=t^3
	\end{align}
	desde $t=0$ a $t=4.$
\end{resuelto}



\begin{resuelto}
	Encuentre la longitud de arco de la cicloide
	\begin{align}
		x = \theta -\sin \theta, y = 1 - \cos \theta
	\end{align}
	entre $\theta = 0$ y $\theta = 2\pi$.

\end{resuelto}



