\section{Bases y dimensi\'on}

\begin{definicion}
 Sea $V$ un espacio vectorial y $B=\set{u_{1},...,u_{k}}\subset V.$ Decimos que $B$ es unconjunto \emph{linealmente
independiente} si para cualesquiera $c_{1},...,c_{k}\in \R,$
$$
c_{1}u_{1}+...+c_{k}u_{k}=0 \Rightarrow c_{1}=...=c_{k}=0,
$$
es decir, la única relaci\'on lineal entre los elementos de $B$ es la trivial. En otro caso, decimos que $B$ es
\emph{linealmente dependiente.}
\end{definicion}

\begin{definicion}
 Decimos que $B\subset V$ es una base de $V$ si:
 \begin{enumerate}
  \item $V=\gen{B}$ y
  \item $B$ es linealmente independiente.
 \end{enumerate}

\end{definicion}

\begin{observacion}
Es decir, $B$ es un base si cualquier $v\in V$ es una combinaci\'on lineal de sus elementos, no falta informaci\'on, y
ninguno de los elementos de la base es combinaci\'on lineal de los restantes, es decir, no sobra informaci\'on. Una vez
que tenemos una base, toda lo que necesitamos saber sobre el espacio vectorial se puede obtener a partir de los
elementos de la base. 
\end{observacion}

\begin{proposicion}
Toda base de un espacio vectorial tiene el mismo número de elementos. 
\end{proposicion}

\begin{definicion}
\begin{enumerate}
 \item  Si $B=\set{u_{1},...,u_{n}}$ es una base de $V,$ decimos que $n$ es la \emph{dimensi\'on} de $V$ y escribimos
 $\dim{V}=n.$
 \item  Si $T:V\to W$ es una transformaci\'on lineal decimos que $\dim(\ker{T})$ es la \emph{nulidad} de $T$ y la
denotamos por $\nul{T}.$
\item Si $T:V\to W$ es una transformaci\'on lineal decimos que $\dim(\Im{T})$ es el rango de $T$ y la
denotamos por $\ran{T}.$
\end{enumerate}
\end{definicion}

Determinar si un conjunto forma una base de $\R^{n}$ puede ser bastante laborioso. Sin embargo, las siguientes dos
proposiciones, que se presentan sin prostración, sirven como criterios avanzados para determinar si un conjuntos es
base.

\begin{proposicion}
\label{prop:1}
 Si $n=\dim{V},$ cualquier conjunto $B\subset V$ linealmente independiente con $n$ elementos es una base de $V.$
\end{proposicion}

\begin{proposicion}
\label{prop:2}
 $B=\set{\bm{a_{1,1}\\ \vdots \\ a_{1,n}},...,\bm{a_{n,1}\\ \vdots \\ a_{n,n}}}$ es un conjunto de vectores linealmente
independientes en $\R^{n}$ si y solo si
$$
\begin{vmatrix}
 a_{1,1} & ... & a_{1,n} \\
 \vdots & & \vdots \\
 a_{n,1} & ... & a_{n,n} \\
\end{vmatrix}
$$
\end{proposicion}

\begin{problema}
 Determine si
  $$B'=\set{\begin{bmatrix}
         1 \\ 0
        \end{bmatrix},
        \begin{bmatrix}
         1 \\ 1
        \end{bmatrix}
}$$ es base de $\R^{2}$.
\end{problema}


\begin{solucion} Sabemos que
 $$B=\set{\begin{bmatrix}
         1 \\ 0
        \end{bmatrix},
        \begin{bmatrix}
         0 \\ 1
        \end{bmatrix}
}$$
es una base de $\R^{2}.$ Entonces $\dim{\R^{2}}=2.$

Pero como
$$\begin{vmatrix}
   1 & 1 \\
   0 & 1
  \end{vmatrix} = 1 \neq 0,
$$ entonces
 $$B'=\set{\begin{bmatrix}
         1 \\ 0
        \end{bmatrix},
        \begin{bmatrix}
         1 \\ 1
        \end{bmatrix}
}$$
es un conjunto de 2 vectores \emph{linealmente independientes}. Por tanto, $B'$ tambi\'en es una base de $\R^{2}.$
\end{solucion}


\begin{problema}
 Encuentre una base para $\ker{T}$ y otra para $\Im{T},$ para la transformaci\'on definida en el ejercicio de muestra
\ref{trans_exmp}. Indique cuál es la dimensi\'on de cada espacio. 
\end{problema}

\begin{solucion}
 Como ya vimos en el ejercicio de muestra \ref{texmp:ker},
  $$
\ker(T)=\gen{\begin{bmatrix}
   -2 \\ 1 \\ 1
  \end{bmatrix}}.
  $$

  Consideremos la ecuaci\'on
  $$
c_{1}\begin{bmatrix}
   -2 \\ 1 \\ 1
  \end{bmatrix}=0,
  $$
es decir $$
\begin{bmatrix}
   -2c_{1} \\ c_{1} \\ c_{1}
  \end{bmatrix}=
  \begin{bmatrix}
   0 \\ 0 \\ 0
  \end{bmatrix}.
  $$
  La única soluci\'on es $c=0$ y por tanto $$
B=\set{\begin{bmatrix}
   -2 \\ 1 \\ 1
  \end{bmatrix}}
  $$
  es un conjunto linealmente independiente.

  Por tanto, $B$ es una base de $\ker{T}$ y $\nul{T}=1.$

De manera similar, en el ejercicio de muestra \ref{texmp:im},
$$
\Im(T)=\gen{\begin{bmatrix}
     1 \\ 0 \\ 2
    \end{bmatrix}, \begin{bmatrix}
         2 \\ -1 \\ 7
        \end{bmatrix}}.
$$

Entonces \begin{align*}
          c_{1}\begin{bmatrix}
     1 \\ 0 \\ 2
    \end{bmatrix}+c_{2} \begin{bmatrix}
         2 \\ -1 \\ 7
        \end{bmatrix}&=
        \begin{bmatrix}
         0 \\ 0\\ 0
        \end{bmatrix}\\
        \Rightarrow
        \begin{bmatrix}
         c_{1}+c_{2} \\ -c_{2} \\ 2c_{1}+7c_{2}
        \end{bmatrix}&=
        \begin{bmatrix}
         0 \\ 0\\ 0
        \end{bmatrix}\\
\Rightarrow c_{1}=c_{2}&=0.
         \end{align*}
         
Por tanto, $$\set{\begin{bmatrix}
     1 \\ 0 \\ 2
    \end{bmatrix}, \begin{bmatrix}
         2 \\ -1 \\ 7
        \end{bmatrix}}$$  es un conjunto linealmente independiente,  y por tanto una base de $\Im(T).$ Entonces
$\ran{T}=2.$  

\end{solucion}

Finalmente, enunciaremos una de las proposicones importantes en nuestro curso. Si $T:V\to W$ es una transformaci\'on
lineal, tenemos ls siguiente relaci\'on entre las dimensiones de $V, \ker{T}$ e $\Im{T}.$
\begin{proposicion}[Teorema de la dimensi\'on]
\label{thm:dim}
$$\dim{V}=\nul{T}+\ran{T}.$$
\end{proposicion}

\subsection*{Ejemplos}

\begin{problema}
 Determine si el conjunto $E$ es base del espacio vectorial $V.$
 \begin{enumerate}
  \item $E=\set{\bm{1\\0},\bm{0\\-1}}$, $V=\R^{2}$,
  \item $E=\set{\bm{1\\0},\bm{1\\1}}$, $V=\R^{2}$,
  \item $E=\set{\bm{1\\0\\0},\bm{1\\1\\0},\bm{1\\1\\1}}$, $V=\R^{3}.$
 \end{enumerate}

\end{problema}


\begin{problema}
 Para cada una de las transformaciones lineales $T:V \to W,$ del ejercicio \ref{exe:trans}, encuentre
 \begin{enumerate}
  \item Una base de $\ker{T},$
  \item Una base de $\Im{T},$
  \item $\nul{T},$
  \item $\ran{T},$
\end{enumerate}
y verifique la afirmaci\'on del teorema \ref{thm:dim}.
\end{problema}







