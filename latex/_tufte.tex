
\usepackage{amsthm,thmtools,xcolor}
\setcounter{tocdepth}{2}
\setcounter{secnumdepth}{2}
\numberwithin{equation}{section}
% https://coolors.co/palette/00111c-001523-001a2c-002137-00253e-002945-002e4e-003356-003a61-00406c
\definecolor{mycolor0}{RGB}{0, 17, 28}
\definecolor{mycolor1}{RGB}{0, 21, 35}
\definecolor{mycolor2}{RGB}{0, 26, 44}
\definecolor{mycolor3}{RGB}{0, 33, 55}
\definecolor{mycolor4}{RGB}{0, 37, 62}
\definecolor{mycolor5}{RGB}{0, 41, 69}
\definecolor{mycolor6}{RGB}{0, 46, 78}
\definecolor{mycolor7}{RGB}{0, 51, 86}
\definecolor{mycolor8}{RGB}{0, 58, 97}
\definecolor{mycolor9}{RGB}{0, 64, 108}

\colorlet{ColorVariable0}{mycolor0}
\colorlet{ColorVariable1}{mycolor1}
\colorlet{ColorVariable2}{mycolor2}
\colorlet{ColorVariable3}{mycolor3}
\colorlet{ColorVariable4}{mycolor4}
\colorlet{ColorVariable5}{mycolor5}
\colorlet{ColorVariable6}{mycolor6}
\colorlet{ColorVariable7}{mycolor7}
\colorlet{ColorVariable8}{mycolor8}
\colorlet{ColorVariable9}{mycolor9}




\titleformat{\chapter}%
{\huge\rmfamily\itshape\color{ColorVariable0}}% format applied to label+text
{\llap{\colorbox{ColorVariable0}{\parbox{1.5cm}{\hfill\itshape\huge\color{white}\thechapter}}}}% label
{2pt}% horizontal separation between label and title body
{}% before the title body
[]% after the title body

\everymath{\color{mycolor0}}
%\everydisplay{\color{red}}

\hypersetup{
	pdftitle={Precálculo},
	pdfauthor={Juliho Castillo Colmenares},
	colorlinks=true,
	linkcolor=ColorVariable9,
	anchorcolor=ColorVariable9,
	runcolor=ColorVariable9,
	filecolor=ColorVariable9,
	citecolor = ColorVariable9,
	urlcolor=ColorVariable9,
	frenchlinks=true
}


% section format
\titleformat{\section}%
{\normalfont\Large\itshape\color{ColorVariable1}}% format applied to label+text
{\llap{\colorbox{ColorVariable1}{\parbox{1.5cm}{\hfill\color{white}\thesection}}}}% label
{1em}% horizontal separation between label and title body
{}% before the title body
[]% after the title body

% subsection format
\titleformat{\subsection}%
{\normalfont\large\itshape\color{ColorVariable8}}% format applied to label+text
{\llap{\colorbox{ColorVariable8}{\parbox{1.5cm}{\hfill\color{white}\thesubsection}}}}% label
{1em}% horizontal separation between label and title body
{}% before the title body
[]% after the title body

\declaretheoremstyle[
headfont=\color{ColorVariable7}\normalfont\bfseries,
bodyfont=\color{ColorVariable2}\normalfont\itshape,
]{colored-27}

\declaretheoremstyle[
headfont=\color{ColorVariable6}\normalfont\bfseries,
bodyfont=\color{ColorVariable3}\normalfont\itshape,
]{colored-36}


\declaretheoremstyle[
headfont=\color{ColorVariable6}\normalfont\bfseries,
bodyfont=\color{ColorVariable3}\normalfont\itshape,
]{colored-45}

\declaretheorem[
style=colored-27,
name=Problema,
numberwithin=chapter,
]{problema}

\declaretheorem[
style=colored-27,
name=Problema Resuelto,
numberwithin=chapter,
]{resuelto}

\declaretheorem[
style=colored-27,
name=Solución,
numberwithin=chapter,
]{solucion}

\declaretheorem[
style=colored-27,
name= Ejemplo,
numberwithin=chapter
]{ejemplo}

\declaretheorem[
style=colored-36,
name= Definición,
numberwithin=chapter,
]{definicion}

\declaretheorem[
style=colored-36,
name= Algoritmo,
numberwithin=chapter,
]{algoritmo}


\declaretheorem[
style=colored-36,
name=Teorema,
numberwithin=chapter,
]{teorema}

\declaretheorem[
style=colored-36,
name=Proposición,
numberwithin=chapter,
]{proposicion}

\declaretheorem[
style=colored-36,
name=Corolario,
numberwithin=chapter,
]{corolario}

\declaretheorem[
style=colored-36,
name=Axioma,
numberwithin=chapter,
]{axioma}

\declaretheorem[
style=colored-36,
name=Tabla de Verdad,
numberwithin=chapter,
]{tdv}

\declaretheorem[
style=colored-45,
name=Observación,
numberwithin=chapter,
]{observacion}

\declaretheorem[
style=colored-45,
name=Sugerencia,
numberwithin=chapter,
]{sugerencia}


