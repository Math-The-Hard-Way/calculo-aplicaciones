\section{Proyecto final: Ecuaciones diferenciales}

\subsection*{Teoría}

Consideremos la siguiente ecuación diferencial
$$
x'(t)=cx(t).
$$
Esta ecuación describe un modelos donde la razón de crecimiento instantaneo $x'$ es propocional al estado del sistema,
en un momento determinado. Aplicaciones de este modelo incluyen:
\begin{enumerate}
 \item Crecimiento poblacional;
 \item decaimiento radioactivo;
 \item la Ley de Newton, para la temperatura de un cuerpos; y
 \item interés compuesto.
\end{enumerate}

De hecho, si conocemos la condición \emph{inicial,} es decir, el valor del sistema en el tiempo $t=0,$ podemos
encontrar una \emph{única solución al problema}.

\begin{thm}
 La única solución \emph{continuamente diferenciable} a la ecuación diferencial
 $$
x'=cx,
 $$
 con condición incial $x(0)=x_{0}$ es
\begin{equation}
 \label{sol:1dim}
 x(t)=e^{tc}x_{0}.
\end{equation}

 \end{thm}

Es fácil comprobar que (\ref{sol:1dim}) es un solución derivando de manera usual; que esta sea la \emph{única solución}
con derivada continua es resultado del \emph{teorema fundamental de las ecuaciones diferenciales ordinarias.}

Sin embargo, este modelo solo describe un sistema con una cantidad que evoluciona con el tiempo, ¿como modelar un
sistema con más cantidades?

Podemos pensar que existen cantidades $x_{1}(t),...,x_{n}(t)$ de manera que la razón de cambio de cada una sea
\emph{combinación lineal} de cada una de los estados del sistema, es decir, para $k=1,...,n$:
$$
x_{k}'(t)=a_{k,1}x_{1}(t)+...+a_{k,n}x_{n}(t).
$$
Esto es una manera de generalizar el hecho de que para una sola cantidad, su razón de cambio instantanea sea
\emph{proporcional}.

De manera matricial, podemos escribir este sistema como
$$
\bm{x_{1}'(t)\\ \vdots \\ x_{n}'(t)}=
\bm{a_{1,1}&...&a_{1,n}\\ \vdots & & \vdots \\ a_{n,1} & ... & a_{n,n}}
\bm{x_{1}(t)\\ \vdots \\ x_{n}(t)}.
$$

Si definimos
$$
\begin{cases}
 x(t)=\bm{x_{1}(t)\\ \vdots \\ x_{n}(t)} \\
 x'(t)=\bm{x_{1}'(t)\\ \vdots \\ x_{n}'(t)} \\
 A=\bm{a_{1,1}&...&a_{1,n}\\ \vdots & & \vdots \\ a_{n,1} & ... & a_{n,n}}  
\end{cases}
$$
el sistema anterior se puede reescribir como
$$
x'(t)=Ax(t).
$$

Note como se parece este sistema al de una sola variable. De hecho, así como podemos definir $e^{a}$ para $a\in \R,$ es
posible definir $e^{A},$ donde $A$ es una matriz $n \times n.$ Para esto, necesitamos la siguiente definición de la
función exponencial.

\begin{defn}
 $$e^{x}=\sum_{k\geq 0} \dfrac{x^{k}}{k!}$$
\end{defn}

Esta definición tiene sentido para matrices porque $A^{k}= A\cdots A$ un número $k$ de veces.

\begin{thm}
 La única solución de la ecuación diferencial \emph{vectorial}
 $$
x'(t)=Ax(t), 
 $$
 para $x(t)\in \R^{n}$ para cada $t\in \R$ y $A\in M_{n},$ con condición inicial
 $$
x_{0}= \bm{x_{1,0} \\ \vdots \\ x_{n,0}} \in \R^{n}
 $$
 es $$
x(t)=e^{tA}x_{0}.
 $$
\end{thm}

Sin embargo, calcular la $n$-ésima potencia de una matriz puede ser demasiado complicado... excepto para matrices
diagonales.

\begin{prop}
 Si $$
D=\bm{\lam_{1} & 0 & & \\ 0 & \lam_{2} & & \\ & & \ddots & \\ &&&\lam_{n}}
 $$
 es una matriz diagonal, entonces
 $$
D^{k}=\bm{\lam_{1}^{k} & 0 & & \\ 0 & \lam_{2}^{k} & & \\ & & \ddots & \\ &&&\lam_{n}^{k}}.
 $$
\end{prop}

\begin{proof} 
 La demostración se puede hacer por inducción. 
\end{proof}

Supongamos que $T:\R^{n}\to \R^{n}$ es una tranformación lineal, cuya representación matricial $A$ (en la base estandar
$E$) es diagonalizable y $P$ es la matriz de paso de la base $F$ de vectores propios a la base $E$. Entonces si $D$ es
la matriz que representa la misma transformación en la base $F$, sabemos que
$$
A= PDP^{-1}.
$$

Por inducción, no es difícil demostrar que
$$
A^{n}=P D^{n} P^{-1},
$$
y por tanto, multiplicando por un escalar $t\in \R$,
$$
tA^{n}=P (tD^{n}) P^{-1}.
$$

Antes de continuar, recordemos que por propiedades distributivas de las matrices
$$
R(M+N)S=RMS+RNS,
$$
o de manera más generalizar $$
\sum\left( RM_{k}S \right)=R\left( \sum M_{k} \right) S. 
$$

Entonces
\begin{align*}
e^{tA}&= \sum_{k\geq 0} \dfrac{(tA)^{k}}{k!} \\
&= \sum_{k\geq 0} \dfrac{\left( P (tD^{n}) P^{-1} \right)^{k}}{k!} \\
&= P \left( \sum_{k\geq 0} \dfrac{\left( tD^{n} \right)^{k}}{k!} \right) P^{-1} \\
&= P e^{tD} P^{-1}.
\end{align*}

Basta entonces encontrar $e^{tD}.$ Pero como vimos, calcular las potencias de $D$ no es dificil.

\begin{align*}
 e^{tD} &= \sum_{k\geq 0} \frac{1}{k!} (tD)^{k} \\
&=\sum_{k\geq 0} \frac{1}{k!}
\bm{(t\lam_{1})^{k} & 0 & & \\ 0 & (t\lam_{2})^{k} & & \\ & & \ddots & \\ &&&(t\lam_{n})^{k}} \\
&= \bm{\sum_{k\geq 0} \frac{1}{k!}(t\lam_{1})^{k} & 0 & & \\ 0 & \sum_{k\geq 0} \frac{1}{k!}(t\lam_{2})^{k} & & \\ & &
\ddots & \\ &&&\sum_{k\geq 0} \frac{1}{k!}(t\lam_{n})^{k}} \\
&= \bm{e^{t\lam_{1}} & 0 & & \\ 0 & e^{t\lam_{2}} & & \\ & & \ddots & \\ &&&e^{t\lam_{n}}}.
\end{align*}

¡Listo!

\subsection*{Ejemplos}

\begin{problema}
 Resuelva el siguiente sistema de ecuaciones diferenciales
 $$\begin{cases}
  x'=-x \\
  y'=x+2y
 \end{cases}$$
 con condiciones inciales
 $$
x(0)=0, \, y(0)=3.
 $$ 
\end{problema}

\begin{solucion}
 Rescribimos $x=x_{1}, y=x_{2}$ y podemos escribir el sistema de forma matricial, en la siguiente manera
 $$
\bm{x_{1}'\\x_{2}'}=
\bm{-1 & 0 \\ 1 & 2} \bm{x_{1}\\x_{2}}.
 $$

 Entonces $$
A=\bm{-1 & 0 \\ 1 & 2}.
 $$

 Usando \texttt{WxMaxima}, podemos encontrar los valores y vectores propios.


\noindent
%%%%%%%%%%%%%%%
%%% INPUT:
\begin{minipage}[t]{8ex}{\color{red}\bf
\begin{verbatim}
(%i1)
\end{verbatim}}
\end{minipage}
\begin{minipage}[t]{\textwidth}{\color{blue}
\begin{verbatim}
A: matrix(
 [-1,0],
 [1,2]
);
\end{verbatim}}
\end{minipage}
%%% OUTPUT:
\definecolor{labelcolor}{RGB}{100,0,0}
\begin{math}\displaystyle
\parbox{8ex}{\color{labelcolor}(\%o1) }
\begin{pmatrix}-1 & 0\cr 1 & 2\end{pmatrix}
\end{math}
%%%%%%%%%%%%%%%


\noindent
%%%%%%%%%%%%%%%
%%% INPUT:
\begin{minipage}[t]{8ex}{\color{red}\bf
\begin{verbatim}
(%i2)
\end{verbatim}}
\end{minipage}
\begin{minipage}[t]{\textwidth}{\color{blue}
\begin{verbatim}
eigenvectors(%);
\end{verbatim}}
\end{minipage}
%%% OUTPUT:
\definecolor{labelcolor}{RGB}{100,0,0}
\begin{math}\displaystyle
\parbox{8ex}{\color{labelcolor}(\%o2) }
[[[-1,2],[1,1]],[[[1,-\frac{1}{3}]],[[0,1]]]]
\end{math}
%%%%%%%%%%%%%%%

Esto quiere decir que $\lam_{1}=-1$ es un valor propio con vector propio $$
\bm{1\\ -\frac{1}{3}},
$$
mientras que $\lam_{2}=2$ también lo es, con vector propio
$$
\bm{0 \\ 1}.
$$

\begin{rem}
Como tenemos dos vectores propios en un espacio de dimensión dos, basta verificar que son linealmente independiente,
para saber que forman una base.  Para comprobar que son linealmente independientes, formamos una matriz que tenga
como columnas a estos vectores y verificamos que esta matriz es invertible.
\end{rem}


\noindent
%%%%%%%%%%%%%%%
%%% INPUT:
\begin{minipage}[t]{8ex}{\color{red}\bf
\begin{verbatim}
(%i4)
\end{verbatim}}
\end{minipage}
\begin{minipage}[t]{\textwidth}{\color{blue}
\begin{verbatim}
P: matrix(
 [1,0],
 [-1/3,1]
);
\end{verbatim}}
\end{minipage}
%%% OUTPUT:
\definecolor{labelcolor}{RGB}{100,0,0}
\begin{math}\displaystyle
\parbox{8ex}{\color{labelcolor}(\%o4) }
\begin{pmatrix}1 & 0\cr -\frac{1}{3} & 1\end{pmatrix}
\end{math}
%%%%%%%%%%%%%%%


\noindent
%%%%%%%%%%%%%%%
%%% INPUT:
\begin{minipage}[t]{8ex}{\color{red}\bf
\begin{verbatim}
(%i5)
\end{verbatim}}
\end{minipage}
\begin{minipage}[t]{\textwidth}{\color{blue}
\begin{verbatim}
determinant(%);
\end{verbatim}}
\end{minipage}
%%% OUTPUT:
\definecolor{labelcolor}{RGB}{100,0,0}
\begin{math}\displaystyle
\parbox{8ex}{\color{labelcolor}(\%o5) }
1
\end{math}
%%%%%%%%%%%%%%%

Entonces, $F=\basis{\bm{1\\ -\frac{1}{3}}, \bm{0 \\ 1}}$ es una base de $\R^{2},$ de vectores propios de $A.$ Por tanto
$A$ es diagonalizable. Como $P$ es la matriz de cambio de la base $F$ a la base estandar $E$, usamos la siguiente
identidad
$$
D=P^{-1}AP,
$$
para encontrar la matriz diagonalizada $D.$ Denotaremos a $P^{-1}$ por $Q.$


\noindent
%%%%%%%%%%%%%%%
%%% INPUT:
\begin{minipage}[t]{8ex}{\color{red}\bf
\begin{verbatim}
(%i6)
\end{verbatim}}
\end{minipage}
\begin{minipage}[t]{\textwidth}{\color{blue}
\begin{verbatim}
Q:invert(P);
\end{verbatim}}
\end{minipage}
%%% OUTPUT:
\definecolor{labelcolor}{RGB}{100,0,0}
\begin{math}\displaystyle
\parbox{8ex}{\color{labelcolor}(\%o6) }
\begin{pmatrix}1 & 0\cr \frac{1}{3} & 1\end{pmatrix}
\end{math}
%%%%%%%%%%%%%%%


\noindent
%%%%%%%%%%%%%%%
%%% INPUT:
\begin{minipage}[t]{8ex}{\color{red}\bf
\begin{verbatim}
(%i7)
\end{verbatim}}
\end{minipage}
\begin{minipage}[t]{\textwidth}{\color{blue}
\begin{verbatim}
D:Q.A.P;
\end{verbatim}}
\end{minipage}
%%% OUTPUT:
\definecolor{labelcolor}{RGB}{100,0,0}
\begin{math}\displaystyle
\parbox{8ex}{\color{labelcolor}(\%o7) }
\begin{pmatrix}-1 & 0\cr 0 & 2\end{pmatrix}
\end{math}
%%%%%%%%%%%%%%%

Entonces, sabemos que
\begin{align*}
 e^{tD}=\bm{e^{-t} & 0 \\ 0 & e^{2t}},
\end{align*}
y podemos hallar $e^{tA}$ con la ecuación
$$
e^{tA}=Pe^{tD}P^{-1}.
$$

Podemos hacer los cálculos en \texttt{WxMaxima} de la siguiente manera


\noindent
%%%%%%%%%%%%%%%
%%% INPUT:
\begin{minipage}[t]{8ex}{\color{red}\bf
\begin{verbatim}
(%i8)
\end{verbatim}}
\end{minipage}
\begin{minipage}[t]{\textwidth}{\color{blue}
\begin{verbatim}
matrix(
 [%e^(-t),0],
 [0,%e^(2*t)]
);
\end{verbatim}}
\end{minipage}
%%% OUTPUT:
\definecolor{labelcolor}{RGB}{100,0,0}
\begin{math}\displaystyle
\parbox{8ex}{\color{labelcolor}(\%o8) }
\begin{pmatrix}{e}^{-t} & 0\cr 0 & {e}^{2\,t}\end{pmatrix}
\end{math}
%%%%%%%%%%%%%%%


\noindent
%%%%%%%%%%%%%%%
%%% INPUT:
\begin{minipage}[t]{8ex}{\color{red}\bf
\begin{verbatim}
(%i9)
\end{verbatim}}
\end{minipage}
\begin{minipage}[t]{\textwidth}{\color{blue}
\begin{verbatim}
P.%o8.Q;
\end{verbatim}}
\end{minipage}
%%% OUTPUT:
\definecolor{labelcolor}{RGB}{100,0,0}
\begin{math}\displaystyle
\parbox{8ex}{\color{labelcolor}(\%o9) }
\begin{pmatrix}{e}^{-t} & 0\cr \frac{{e}^{2\,t}}{3}-\frac{{e}^{-t}}{3} & {e}^{2\,t}\end{pmatrix}
\end{math}
%%%%%%%%%%%%%%%

Es decir,
$$
e^{tA}=\begin{bmatrix}{e}^{-t} & 0\cr \frac{{e}^{2\,t}}{3}-\frac{{e}^{-t}}{3} & {e}^{2\,t}\end{bmatrix}
$$

Las condiciones inciales se pueden escribir en forma vectorial como
$$
\bm{0 \\ 3},
$$
y por tanto, nuestra solución sera
$$
\begin{bmatrix}{e}^{-t} & 0\cr \frac{{e}^{2\,t}}{3}-\frac{{e}^{-t}}{3} & {e}^{2\,t}\end{bmatrix}\bm{0 \\ 3}.
$$

Realizamos los cálculos en \texttt{WxMaxima} de la siguiente manera. Primero introducimos el vector como si fuera una
matriz de dos reglos y una columnas


\noindent
%%%%%%%%%%%%%%%
%%% INPUT:
\begin{minipage}[t]{8ex}{\color{red}\bf
\begin{verbatim}
(%i10)
\end{verbatim}}
\end{minipage}
\begin{minipage}[t]{\textwidth}{\color{blue}
\begin{verbatim}
matrix(
 [0],
 [3]
);
\end{verbatim}}
\end{minipage}
%%% OUTPUT:
\definecolor{labelcolor}{RGB}{100,0,0}
\begin{math}\displaystyle
\parbox{8ex}{\color{labelcolor}(\%o10) }
\begin{pmatrix}0\cr 3\end{pmatrix}
\end{math}
%%%%%%%%%%%%%%%

y posteriormente hacemos la multiplicación, recordando que \texttt{WxMaxima} le asigno la etiqueta \texttt{\%o9} a
nuestra matriz $e^{tA}$, y la etiqueta \texttt{\%o10} a nuestro vector de condiciones inciales.


\noindent
%%%%%%%%%%%%%%%
%%% INPUT:
\begin{minipage}[t]{8ex}{\color{red}\bf
\begin{verbatim}
(%i11)
\end{verbatim}}
\end{minipage}
\begin{minipage}[t]{\textwidth}{\color{blue}
\begin{verbatim}
%o9.%o10;
\end{verbatim}}
\end{minipage}
%%% OUTPUT:
\definecolor{labelcolor}{RGB}{100,0,0}
\begin{math}\displaystyle
\parbox{8ex}{\color{labelcolor}(\%o11) }
\begin{pmatrix}0\cr 3\,{e}^{2\,t}\end{pmatrix}
\end{math}
%%%%%%%%%%%%%%%

Por tanto, la solucion a nuestro sistema de ecuaciones diferenciales es
$$
\bm{x(t)\\y(t)}=
\bm{0 \\ 3 e^{2t}}.
$$

\end{solucion}

\subsection*{Proyecto final}

Resuelva los siguientes sistema de ecuaciones diferenciales, como se hizo en el ejemplo anterior. Debe plantear de
manera correcta todos los pasos, indicar los cálculos que hizo en \texttt{WxMaxima} y escribiendo de manera clara sus
conclusiones. El proyecto puede ser elaborado por equipos de a lo más tres personas, y debe ser entregado en
computadora el día del examen final.\footnote{Para copiar el cógido que introduzca en \texttt{WxMaxima}, seleccione con
el botón izquierdo de su ratón, el lado
izquierdo del código, de manera que cambie a color azul como en la figura \ref{fig:1} y posteriormente presione el
botón derecho, y seleccione la opción copiar. }

% \begin{figure}[t]
%  \centering
%  \includegraphics[height=5cm,viewport=0 0 1024 576,keepaspectratio=true]{./pantalla.png}
%  % pantalla.png: 1366x768 pixel, 96dpi, 36.14x20.32 cm, bb=0 0 1024 576
%  \caption{WxMaxima}
%  \label{fig:1}
% \end{figure}

\begin{enumerate}
 \item
 $$
\begin{cases}
 x_{1}'=2x_{1}+x_{2} \\
 x_{2}'=x_{1} + x_{2}
\end{cases}
 $$
con condiciones iniciales $x_{1}(0)=1, \, x_{2}(0)=1.$

\item
$$x'=Ax$$ con
$$
A=\bm{0 & 3 \\ 1 & -2}
$$
y condiciones iniciales
$$
x(0)=\bm{3 \\ 1}.
$$

\item
$$
x'=Ax
$$
con
$$
A=\bm{2&0&0\\0&-1&0\\0&2&-3}
$$
y condiciones iniciales
$$
x(0)=\bm{0\\-b\\b}.
$$
\end{enumerate}

%Para una revisión con más detalle de este tema y un repaso de álgebra lineal, puede consultar \cite[capítulo 3]{HS}.
