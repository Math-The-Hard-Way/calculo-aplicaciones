
%403
\section{Determinantes}

\subsection{Determinantes de Segundo Orden}

{Definición}
	\begin{definicion}
		$$
		\begin{vmatrix}
			a & b \\ c & d
		\end{vmatrix}=ad-bc.
		$$
	\end{definicion}
	



	\begin{problema}
		$$
		\begin{vmatrix} 2 & 3 \\ -1 & -2 \end{vmatrix}=
		$$
	\end{problema}
	



	Si consideremos el siguiente sistema de ecuaciones
	\[
		\begin{cases}
			a_{1}x+b_{1}y=c_{1}\\
			a_{2}x+b_{2}y=c_{2}
		\end{cases}...
	\]
	



	...y definimos
	\begin{align*}
		\Delta&=\begin{vmatrix} a_{1} & b_{1} \\ a_{2} & b_{2} \end{vmatrix}\\
		\Delta_{x}&=\begin{vmatrix} c_{1} & b_{1} \\ c_{2} & b_{2} \end{vmatrix}\\
		\Delta_{y}&=\begin{vmatrix} a_{1} & c_{1} \\ a_{2} & c_{2} \end{vmatrix}...
	\end{align*}



	...entonces
	\[
		\label{spi:28.2}
		\begin{split}
			x&=\dfrac{\Delta_{x}}{\Delta}\\
			y&=\dfrac{\Delta_{y}}{\Delta}
		\end{split}
	\]
	



	\begin{problema}
		Resuelva el sistema
		$$\begin{cases}
			2x+3y=8\\
			x-2y=-3
		\end{cases}
		$$
	\end{problema}
	

\subsection{Ejemplos}


	El m\'etodo de soluci\'on de sistemas de ecuaciones linales, por medio de determinantes, se conoce como Regla de Cramer.



	\begin{problema} Resuelva el siguiente sistema por la Regla de Cramer
		\label{spi:28.4a}
		$$
		\begin{cases}
			4x+2y=5\\
			3x-4y=1
		\end{cases}
		$$
	\end{problema}
	



	\begin{problema} Resuelva el siguiente sistema por la Regla de Cramer
		\label{spi:28.4b}
		$$
		\begin{cases}
			3u+2v=18\\
			-5u-v=12
		\end{cases}
		$$
	\end{problema}
	


\subsection{Sistemas Indeterminados}


	Un sistema de $ n $ ecuaciones con $ n $ incognitas tiene una única solución si y solo si su determinante principal $ \Delta \neq 0. $  
	
	En este caso, decimos que el sistema es consistente.



	Si $ \Delta =0 $, entonces o bien existen multiples soluciones, o bien no existe alguna en absoluto.
	
	En cualquier caso, decimos que el sistema es inconsistente.



	Determine si 
	$$\begin{cases}
		5x-2y=10\\
		10x-4y=20
	\end{cases}$$
	es consistente; y de no ser el caso, explique que sucede con las soluciones.



	Determine si 
	$$\begin{cases}
		5x+3y=15\\
		10x+6y=60
	\end{cases}$$
	es consistente; y de no ser el caso, explique que sucede con las soluciones.




\subsection{Ejemplos}


	\begin{problema}
		\begin{align*}
			2x-y&=4\\
			x+y&=5
		\end{align*}
		
	\end{problema}
	



	\begin{problema}
		\begin{align*}
			5x+2y&=3\\
			2x+3y&=-1
		\end{align*}
		
	\end{problema}
	



	\begin{problema}
		\begin{align*}
			2x+3y=3\\
			6y-6x=1
		\end{align*}
		
	\end{problema}
	



	\begin{problema}
		\begin{align*}
			5y&=3-2x\\
			3x&=2y+1
		\end{align*}
		
	\end{problema}
	



	\begin{problema}
		\begin{align*}
			\dfrac{x-2}{3}+\dfrac{y+1}{6}=2\\
			\dfrac{x+3}{4}-\dfrac{2y-1}{2}=1
		\end{align*}
		
	\end{problema}
	

