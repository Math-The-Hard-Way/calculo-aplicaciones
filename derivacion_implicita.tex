
\section{Derivación implícita}



    Denotaremos por $y'$ la derivada $\frac{dy}{dx}.$
    
    \begin{problema}
        Si $$x^{2}+y^{2}=r^{2},$$
        donde $r$ es una constante, encontrar $y'.$
    \end{problema}


    \begin{sol}
        Por regla de la cadena, $(y^{2})'=2yy'.$ Esto porque
        $$
        \dfrac{dy^{2}}{dx}=\dfrac{dy^{2}}{dy}\dfrac{dy}{dx}.
        $$
        Si derivamos el lado izquierdo de la ecuación, respecto de $x,$ usando linealidad, obtenemos
        $
        2x+2yy',
        $
        mientras que si derivamos el derecho, ya que $r^{2}$ es contante, obtenemos cero e igualando, tenemos que
        $$
        2x+2yy'=0.
        $$
        Después de despejar obtenemos que $$
        y'=-\dfrac{x}{y}.
        $$
    \end{sol}



    \begin{rem}
        Observe que $$x^{2}+y^{2}=r^{2}$$ es la ecuación de un círculo con centro en el origen con radio $r>0.$ Use \texttt{Sagemath}
        para graficar esta ecuación para un radio dado, por ejemplo, $r=5.$
        \begin{enumerate}
            \item Compare las pendientes de las rectas tangente en $\left( x,y \right)$ y $\left( -x,-y \right).$ ¿Que
            relación sobre estás dos rectas podemos deducir?
            \item Compare las pendiente de la recta tangente en $\left( x,y \right)$ y la recta que pasa por el origen y este punto.
            ¿Que
            relación sobre estás dos rectas podemos deducir?
        \end{enumerate}
    \end{rem}



    \begin{problema}
        Si $$x^{3}+y^{3}=6xy,$$ encontrar $y'.$
    \end{problema}


    \begin{sol}
        Por regla de la cadena
        $$
        \dfrac{d}{dx}\left( y^{3} \right)=\dfrac{dy^{3}}{dy}\dfrac{dy}{dx},
        $$
        
        es decir, $(y^{3})'=\left( 3y^{2} \right)\left( y' \right).$
        
        Además, por la regla de Leibniz,
        $$
        (xy)'=x'y+xy'=y+xy'.
        $$
        
        Entonces, derivando ambos lados de la ecuación, y usando linealidad, tenemos que
        $$
        3x^{2}+3y^{2}y'=6\left(y+xy'  \right).
        $$
        
        Despejando $y',$ obtenemos
        $$
        y'=\dfrac{2y-x^{2}}{y^{2}-2x}.
        $$
    \end{sol}


    \begin{problema}
        Encuentre $y'$ en términos de $x$, si $y=\arcsin(x),$ con imagen $-\frac{\pi}{2}\leq y \leq \frac{\pi}{2}.$
    \end{problema}

{Solución}
    En este caso $x=\sin(y).$ Por regla de la cadena obtenemos que
    $$
    \dfrac{d}{dx}\sin(y)=\dfrac{d}{dy}\sin(y)\dfrac{dy}{dx}=\cos(y)y'.
    $$



    Sin embargo,también sabemos que
    $$
    \dfrac{d}{dx}\sin(y)=\dfrac{dx}{dx}=1.
    $$



    Por lo cual $1=\cos(y)y',$ y entonces $y'=1/\cos(y).$ Pero también sabemos que, por la manera en que escogemos el
    rango de $y$, $\cos(y)>0,$ y por tanto
    $$
    \cos(y)=\sqrt{1-\sin^{2}(y)}=\sqrt{1-x^{2}}.
    $$



    Es decir
    $$
    y'=\dfrac{1}{\sqrt{1-x^{2}}}.
    $$
    
    



    % Para resolver los siguientes ejercicios, puede consultar los ejemplos de \cite[sec. 3.5]{S}.
    
    \begin{problema} Encuentre $y'.$  %Grafique $y$  y $y'$ usando \texttt{Sagemath}.
        \begin{enumerate}
            \item $\sin(x+y)=y^{2}\cos(x),$
            \item $x^{4}+y^{2}=16,$
            \item $y=\arccos(x),$
            \item $y=\arctan(x).$
        \end{enumerate}
    \end{problema}



