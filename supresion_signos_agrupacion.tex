
%202
\section{Supresi\'on de signos de agrupaci\'on}


	Cuando queremos quitar parentesis u otro signo de agrupaci\'on, en una suma o resta de polinomios, basta usar la regla de los signos.



	Sin embargo, cuando un polinomio se multiplica por un coeficiente, se utiliza la siguiente regla
	\begin{prop}[Ley de la distribuci\'on]
		\begin{eqnarray}
			\label{distribucion}
			a\left( b+c \right)=ab+ac\\
			\left( a+b \right)\left( c+d \right)=ac+ad+bc+bd.
		\end{eqnarray}
		
	\end{prop}
	



	\begin{problema} Simplifique
		\begin{enumerate}
			\item $(6-7a)(2-4a)$
			\item $-4(-4w-5)(4w-2)$
			\item $6(-5v-7)(4-5v)(-2v-1)v$
		\end{enumerate}	
	\end{problema}
	

