%103
\section{Razones y proporciones}

{Proporciones entre números enteros}
	La raz\'on de dos números (enteros o racionales) $a,b$ se escribe $a:b$ y se representa por la fracci\'on $\dfrac{a}{b}.$
	
	\begin{problema}
		La raz\'on de 4 a 6 se escribe $4:6$ y se representa por la fracci\'on $\dfrac{4}{6}=\dfrac{2}{3}.$ 
	\end{problema}
	
	


{Proporciones entre fracciones}
	\begin{problema}
		La raz\'on de $\frac{2}{3}$ a $\frac{4}{5}$ se escribe $$\dfrac{2}{3}:\dfrac{4}{5}$$ y se representa por la fracci\'on
		$$
		\dfrac{\frac{2}{3}}{\frac{4}{5}}
		=\dfrac{2}{3}\div\dfrac{4}{5}=\dfrac{5}{6}.
		$$
	\end{problema}
	



	Diremos que dos razones $a:b$ y $c:d$ son equivalente si 
	$$\dfrac{a}{b}=\dfrac{c}{d}.$$
	
	En ese caso, escribimos $a:b \sim c:d.$



	\begin{problema}
		¿Cuál es el precio unitario de cada art\'iculo?
		\begin{itemize}
			\item Una bote con 3 litros de aceite cuesta \$54.
			\item Una caja de cereales con 700 gramos cuesta \$63.
		\end{itemize}
		
	\end{problema}
	



	\begin{problema}
		Exprese las siguientes razones por medio de una fracci\'on simplificada
		\begin{enumerate}
			\item $96:128$
			\item $\frac{2}{3}:\frac{3}{4}$
		\end{enumerate}
		
	\end{problema}
	


% 
% 	\begin{problema}
% 		Encuentre la raz\'on entre las siguientes cantidades:
% 		\begin{enumerate}
% 			\item 6 libras a 12 onzas
% 			\item 3 cuartos a 2 galones
% 			\item 3 yardas a 6 pies cuadrados
% 		\end{enumerate}
% 		
% 	\end{problema}
% 	
% 


	\begin{problema}
		Un segmento de 30 pulgadas se divide en dos partes cuyas longitudes están en raz\'on de $2:3.$ Encuentre las longitudes de ambos segmentos.
	\end{problema}
	



	\begin{problema}
		Las edades actuales de dos hermanos son 5 y 8 años respectivamente. ¿Al cabo de cuantos años, sus edades estarán en raz\'on $3:4$?
	\end{problema}
	



	\begin{problema}
		Divida 253 en cuatro partes propocionales $2:5:7:9.$
	\end{problema}
	


\subsection{Razones inversas}


	Cuando tratamos de conservar una proporci\'on $a:b,$ hablamos de una \emph{raz\'on directa,} y podemos representarla por una equivalencia de fracciones:
	$$
	a:b \sim c:d \Leftrightarrow
	\dfrac{a}{b} = \dfrac{c}{d}.
	$$



	Por ejemplo, en una recete de hotcakes, tenemos una raz\'on
	$1:\frac{3}{4}$ entre tazas de harina y tazas de leche.
	



	Para mantener la receta, podemos multiplicar las cantidades, pero manteniendo la proporci\'on. 
	



	Por ejemplo, podemos ocupar 4 tazas de harina para 3 tazas de leche, porque 
	$1:\frac{3}{4} \sim 4:3.$



	En cambio, en ocasiones lo que buscamos es mantener una cantidad total, y no proporci\'on. Generalmente, es una cantidad de trabajo.



	Por ejemplo, considere el trabajo de pintar una pared de dimensiones fijas.
	Supongamos que una persona puede pintarla en 8 horas; pero suponiendo que contratamos un pintor más con una experiencia similar, 
	\begin{enumerate}
		\item ¿cuantas horas se requerirán para terminar el trabajo? 
		\item ¿Y si contratáramos 4 pintores? 
		\item ¿Y si fueran 8?
	\end{enumerate}
	



	En el ejemplo anterior, la pared requiere \emph{8 horas-trabajador;} esta es la cantidad que debemos conservar, aunque no es permitido variar los trabajadores o las horas de trabajo por trabajador.



	En este caso, hablamos de una \emph{raz\'on inversa.} 



	\begin{problema}
		Sabiendo que 8 personas tardan 12 d\'ias en poner a punto 16 maquinas, encuentre el número de d\'ias que emplearán 16 personas en poner a punto 8 máquinas. 
	\end{problema}
	



	\begin{problema}
		Sabiendo que 8 personas tardan 12 d\'ias en poner a punto 16 maquinas, encuentre el número de d\'ias que emplearán 15 personas en poner a punto 50 máquinas. 
	\end{problema}
	


\subsection{Ejemplos}


	\begin{problema}
		¿Cuál es la mejor compra entre 7 latas de sopas que cuestan \$22.50 y 3 latas del mismo producto, que cuestan \$9.50.
	\end{problema}
	



	\begin{problema}
		¿Cuál es la mejor compra entre un paquete de 3 onzas de queso crema que cuesta \$4.30 y otro paquete de 8 onzas del mismo producto que cuesta \$8.70?
	\end{problema}
	



	\begin{problema}
		Si dos hombres pueden transportar 6 acres de tierra en 4 horas, ¿cuántos hombres se necesitan para transportar 18 acres en 8 horas?
	\end{problema}
	

%%%%%%%%%%%%%%%%%%%%%
{}
	\begin{problema}
		Resuelva la proporción
		\begin{align*}
			\dfrac{x}{63}=\dfrac{5}{9}
		\end{align*}
		
	\end{problema}
	

%%%%%%%%%%%%%%%%%%%%%
{}
	\begin{problema}
		Resuelva la siguiente ecuación usando productos cruzados
		\begin{align*}
			\dfrac{x-2}{5}=\dfrac{x+1}{3}
		\end{align*}
		
	\end{problema}
	

%%%%%%%%%%%%%%%%%%%%%
{}
	\begin{problema}
		Enfermeras usan proporcoines para determinar la cantidad de medicina a administrar, cuando la dosis es medida en miligramos \texttt{(mg)}, pero la medicina es empacada en una forma diluida en milímetros \texttt{(mL)}.	
		
		
		
		Por ejemplo, para encontrar los mililitros de fluido necesario para administrar \texttt{300mg} de un medicamento de una medicina que viene empacada como \texttt{120mg} en \texttt{2mL} de un fluído, se plantean la proporción
		\begin{align*}
			\dfrac{\texttt{120mg}}{\texttt{2mL}}=\dfrac{\texttt{300mg}}{x \texttt{ mL}}
		\end{align*}
		donde $x$ representa la cantidad a administrar en mililitros. 
		
		Resuelva la proporción anterior.
	\end{problema}

